\chapter{做市收益的来源与均值回复结构}
\label{chap:profit_ou}

\section{引言}

在前面几章中,我们基于 Avellaneda--Stoikov(AS)与 Guéant--Lehalle--Tapia(GLT)模型构建了最优做市的数学框架。然而一个更基础的问题是:

\begin{center}
\textit{做市为什么能赚钱?在什么条件下赚钱?在什么条件下会亏钱?}
\end{center}

这一问题不仅是理论上的,也是实际交易系统设计最核心的部分。在高频环境中,做市盈利主要来自三类因素:

\begin{enumerate}
  \item \textbf{spread capture(赚价差)}
  \item \textbf{inventory PnL(库存随价格变化带来的盈亏)}
  \item \textbf{order-flow prediction(短期价格预测)}
\end{enumerate}

其中第二项始终是负值(价格有噪声,库存引入风险),因此做市收益必须依赖于:

\[
\textbf{spread capture} > \textbf{inventory risk} + \textbf{adverse selection}.
\]

Chakraborty--Kearns(2011)\cite{ChakrabortyKearns2011} 提供了一个十分优雅的框架,把做市商的长期收益拆解为:

\begin{equation}
\mathbb{E}[\text{PnL}]
=
\underbrace{\mathbb{E}[\text{spread capture}]}_{\text{正收益}}
-
\underbrace{\mathbb{E}[q_t\, \mathrm{d}S_t]}_{\text{inventory risk}}
-
\underbrace{\text{adverse selection}}_{\text{信息不对称成本}},
\label{eq:pnl_decomp}
\end{equation}

并证明在一个均值回复价格过程下,做市策略可以长期正收益;而在强趋势市场中,做市必然亏损。

本章将从数学和经济学两方面深入分析做市盈利的来源。

\section{做市的 PnL 分解公式}

回顾总资产:

\[
W_t = X_t + q_t S_t.
\]

微分:

\[
\mathrm{d}W_t 
= \mathrm{d}X_t + q_t \mathrm{d}S_t.
\]

其中:

- $\mathrm{d}X_t$ 来自买卖价差(spread capture)
- $q_t \mathrm{d}S_t$ 来自库存风险(inventory PnL)

做市商的交易收益来自被动成交:

\[
\mathrm{d}X_t
=
(S_t + \delta_t^a)\,\mathrm{d}N_t^a
-
(S_t - \delta_t^b)\,\mathrm{d}N_t^b,
\]

其中 $\mathrm{d}N_t^{a,b}$ 是成交点过程。

取期望可得:

\[
\mathbb{E}[\mathrm{d}W_t]
=
\underbrace{
\delta_t^a \lambda_t^a + \delta_t^b \lambda_t^b
}_{\text{spread capture}}
+
\underbrace{
q_t\, \mu_t
}_{\text{inventory drift term}}.
\]

在均值为零的价格模型(如 $\mu_t=0$)下,第二项为零,但库存的方差导致风险成本:

\[
\mathrm{Var}[\mathrm{d}W_t] = q_t^2 \sigma^2.
\]

因此做市商必须控制库存波动,并以较大的价格偏移对冲风险,这正是 AS/GLT 模型中 reservation price 的来源。

\section{基于价格过程的长期盈利条件}

Chakraborty--Kearns(2011)做了一个重要发现:  
\textbf{做市商的长期收益完全取决于价格过程是否均值回复。}

他们考虑一个简单的做市策略:

- 每当价格上升到上轨,做市商卖出;
- 每当价格下降到下轨,做市商买入;
- 目标是赚取均值回复带来的回调。

假设价格满足 Ornstein--Uhlenbeck(OU)过程:

\[
\mathrm{d}S_t = \kappa(\theta - S_t)\,\mathrm{d}t + \sigma\,\mathrm{d}W_t.
\]

则做市商盈利的必要条件是:

\[
\kappa > 0.
\]

即价格对均值的回归速度为正。

进一步可得做市长期平均收益:

\[
\text{PnL} 
\approx 
2\Delta \cdot \kappa (S_t - \theta)
-
\frac{1}{2}\sigma^2 \Delta^2,
\]

其中 $\Delta$ 是做市商设置的对称挂单距离。

取期望,可得长期正收益条件:

\[
2\kappa \mathbb{E}[|S_t-\theta|] 
>
\frac{1}{2}\sigma^2 \Delta.
\]

该式意义明确:

- \textbf{均值回复越强} → 做市越赚钱;
- \textbf{波动越大} → 风险越大,spread 必须变宽;
- \textbf{挂单距离太宽} → 参与市场太少,spread capture 下降。

\section{为什么均值回复带来做市盈利?}

直觉如下:

- 做市商的库存倾向在价格上升时为负(卖出更多)
- 在价格下降时为正(买入更多)

因此库存与价格负相关:

\[
\text{Corr}(q_t, S_t) < 0.
\]

如果价格会回到均值,则:

- 高价卖出的库存会在低价回补 → 赚价差
- 低价买入的库存会在高价卖出 → 赚价差

这是 \textbf{inventory mean reversion}。

数学上:

\[
\mathbb{E}[q_t\, \mathrm{d}S_t] 
< 0
\quad\Longleftrightarrow\quad
\text{有盈利}.
\]

反之,如果价格有正趋势($\kappa <0$ 或 $\mu>0$):

\[
q_t \text{ 在趋势方向累积 } \Rightarrow \text{亏损}.
\]

这正解释了:

- crypto 上涨周期中做市商普遍亏损;
- 横盘震荡时期做市商利润极高。

\section{做市的收益结构:Adverse Selection 与 Price Impact}

完整 PnL 分解如下:

\begin{equation}
\mathrm{d}W_t
=
\underbrace{\text{spread capture}}_{(\delta^a\lambda^a + \delta^b\lambda^b)\,\mathrm{d}t}
\quad
-
\underbrace{q_t\,\mathrm{d}S_t}_{\text{inventory risk}}
\quad
-
\underbrace{\text{price impact}}_{\text{adverse selection}}.
\label{eq:pnl_full}
\end{equation}

其中 adverse selection 来自:

- 市价单方向 → 短期价格变化方向  
- 做市商被动吃单 → 往往在错误一侧成交  

若市价买单预示价格上行,则:

做市商的 ask 成交后  
→ 价格继续上涨  
→ 做市商亏钱。

做市要盈利必须要求:

\[
\text{spread capture}
>
\text{adverse selection}.
\]

而改进做法是提升预测水平:

- 若能预测订单流(例如 Hawkes 过程),可以调整挂单位置;
- 若能预测微价格(Stoikov 2018),可提前撤单或换档挂单。

\section{Crypto CEX 市场中的均值回复与做市盈利性}

加密货币市场展现了与传统股票不同的均值回复结构:

\subsection{短期均值回复强:}

来自:

- CEX order-book resiliency(Large 2007)
- 永续合约–现货基差收敛机制(funding)
- 内嵌的清算与杠杆机制
- 多交易所交叉影响(cross-exchange impact)

因此:

\[
\textbf{1-5 秒时间尺度上价格强烈均值回复}.
\]

这使得高频做市极其盈利。

\subsection{中长期趋势强:}

来自:

- 宏观资金面;
- BTC/ETH 作为风险资产的供需变化;
- 链上结构性因素(矿工、stakers)。

因此:

\[
\textbf{数小时–数天尺度上价格具有强趋势}.
\]

这解释了:

- 高频做市盈利稳定;
- 中频做市(如 5–15 分钟)常常因为趋势而亏损;
- 需要 GLT 模型强库存约束。

\section{做市盈利条件总结}

基于以上推导,做市盈利的必要条件可归纳为:

\begin{enumerate}
  \item \textbf{价格存在短期均值回复}  
      \[
      \kappa > 0 \quad\text{或}\quad
      \text{Corr}(q_t,\mathrm{d}S_t)<0.
      \]

  \item \textbf{订单流与价格没有强负向预测性}  
      (否则 adverse selection 过高)

  \item \textbf{库存风险可控}  
      (必须依赖 GLT 库存约束或动态对冲)

  \item \textbf{spread 足够覆盖价格噪声}  
      \[
      \delta > \sigma \sqrt{\mathrm{d}t}.
      \]

  \item \textbf{流动性足够高(成交概率高)}  
      ($A,k$ 估计稳定)
\end{enumerate}

在 crypto CEX 中,这五条都成立,因此做市长期盈利强。

\section{本章小结}

本章从 PnL 分解出发,揭示了做市盈利的本质来源:

\begin{itemize}
  \item spread capture(赚价差)是主要正收益;
  \item inventory risk 与 adverse selection 是主要负收益;
  \item OU 模型中的均值回复是盈利关键;
  \item Crypto CEX 在极短时间尺度具有非常强的均值回复,使得高频做市特别赚钱。
\end{itemize}

下一章将引入更多现实因素(drift、jump、Hawkes、手续费结构等),构建面向 Crypto 做市的扩展模型族。