\chapter{做市模型的扩展:订单流自激性、价格冲击、费用结构与跳跃}
\label{chap:extensions}

\section{引言}

在前几章中,我们重点讨论了 Avellaneda--Stoikov(AS)与 Guéant--Lehalle--Tapia(GLT)模型,它们在实际做市和学术界具有基础性地位。但这些模型的原始设定具有以下限制:

\begin{enumerate}
  \item 订单流为简单的 Poisson 模型,而真实订单流呈现自激性和簇集性;
  \item 价格模型为布朗运动,而 crypto CEX 中存在跳跃与冲击;
  \item 没有考虑手续费结构(maker rebate),实际收益偏差较大;
  \item 没有考虑永续合约特有的 funding rate 驱动的 drift;
  \item 库存管理没有考虑多交易所结构的基差风险;
  \item 没有加入订单簿状态的信息,例如 microprice 和 imbalance;
  \item 没有考虑批量成交、大单冲击与清算事件的影响;
\end{enumerate}

因此,本章将在 AS/GLT 框架基础上,引入更符合真实 crypto 市场的扩展模型族。

\section{带自激性订单流的 Hawkes 做市模型}

\subsection{Hawkes 过程回顾}

Bacry--Muzy(2015)\cite{BacryMuzyReview2015} 证明金融订单到达过程高度自激,可以表示为 Hawkes 过程:

\[
\lambda_t
=
\mu
+
\sum_{t_i<t}
\alpha e^{-\beta(t - t_i)}.
\]

其中:

- $\alpha$: 自激强度  
- $\beta$: 衰减速度  
- $\mu$: 基础流动性  

条件期望:

\[
\mathbb{E}[\lambda_t] = \frac{\mu}{1 - \alpha/\beta}.
\]

因此当 $\alpha/\beta$ 接近 1 时:

- 市场出现“流动性挤兑”(resiliency 下降)
- 做市风险急剧上升
- spread 应大幅扩张

\subsection{在做市 HJB 中加入 Hawkes 的方法}

对于 bid 与 ask 两侧:

\[
\lambda_t^b = A_b e^{-k_b \delta_t^b}
+
\sum_{t_i<t} \alpha_b e^{-\beta_b (t - t_i)},
\]

\[
\lambda_t^a = A_a e^{-k_a \delta_t^a}
+
\sum_{t_i<t} \alpha_a e^{-\beta_a (t - t_i)}.
\]

代入 AS 模型中的 HJB:

\[
\partial_t u + 
\frac{1}{2}\sigma^2 \partial_{ss} u
+
\max_{\delta^b} \lambda_t^b(\delta^b) \Delta u_{q+1}
+
\max_{\delta^a} \lambda_t^a(\delta^a) \Delta u_{q-1}
= 0.
\]

此处的主要修改是:

- 最优报价 $\delta^*$ 不再只依赖库存,还依赖 \(\lambda_t^b,\lambda_t^a\)
- 订单流自激性越强 → 越需要扩大报价距离以减少 adverse selection

这是 crypto CEX 做市中最重要的扩展之一。

\section{加入价格冲击(Market Impact)的做市模型}

\subsection{瞬时冲击与永久冲击}

Gatheral(2010)提出了 no-dynamic-arbitrage 的冲击结构:

\[
S_{t^+} = S_t + \underbrace{\eta v_t}_{\text{instantaneous}}
+ \underbrace{\int_0^\infty G(u)v_{t-u}\,\mathrm du}_{\text{transient}},
\]

其中:

- $v_t$ 为交易速度
- $\eta$ 为瞬时冲击系数
- $G(u)$ 为冲击核(impact kernel)

如果允许做市商主动调整库存(maker-taker hybrid),冲击不可忽略。

在做市框架中,只需修正:

- ask 成交通常意味着价格向上冲击  
- bid 成交通常意味着价格向下冲击  

因此 $\mathrm{d}S_t$ 应加入:

\[
\mathrm{d}S_t = \sigma \mathrm{d}W_t + \eta (\mathrm{d}N_t^a - \mathrm{d}N_t^b).
\]

带入 AS/GLT 模型,会发现:

- 最优 spread 必须随 $\eta$ 增大;
- 市价单方向越强 → 做市商越容易亏损 → 扩大价差;
- 使 adverse selection 提升为显性风险项。

\section{加入 maker rebate / taker fee 的有效价差模型}

Crypto CEX 通常有以下费用结构:

- maker:$r_m < 0$(负费 = rebate)
- taker:$f_t > 0$

做市商挂单成交后收益变为:

\[
\delta^{b,eff}
=
\delta^b + r_m,
\qquad
\delta^{a,eff}
=
\delta^a + r_m.
\]

因此 AS/GLT 中最优价差应替换为:

\[
\delta^{b,eff}(q), \quad \delta^{a,eff}(q).
\]

若 $|r_m|$ 较大(如 Binance 期货 VIP),可能出现:

- “zero-spread market making”
- 甚至 “negative spread arbitrage”(即补贴做市)

这是 crypto CEX 中做市利润较高的主要原因之一。

\section{加入跳跃过程(Jump Diffusion)}

Crypto 价格常出现跳跃:清算、监管、链上事件。

模型:

\[
\mathrm{d}S_t
=
\sigma \mathrm{d}W_t
+
\kappa \mathrm{d}J_t,
\quad
\mathbb{P}(\mathrm{d}J_t=1)=\lambda_J\,\mathrm{d}t.
\]

做市商在跳跃风险下应:

\begin{itemize}
  \item 扩大价差;
  \item 降低库存上限;
  \item 增大风险厌恶参数 $\gamma$;
\end{itemize}

跳跃会改变 AS 模型的 reservation price:

\[
r_t = S_t - q_t \gamma (\sigma^2 + \lambda_J \kappa^2)(T-t).
\]

库存风险变大。

\section{加入永续合约 funding rate 的漂移项}

永续合约价格近似满足:

\[
\mathrm{d}S_t = f_t S_t\,\mathrm{d}t + \sigma \mathrm{d}W_t.
\]

资金费率 $f_t$ 会带来方向性风险:

- $f_t > 0$ → 永续价格向上漂移 → 做市商容易积累空头亏损
- $f_t < 0$ → 永续价格向下漂移 → 做市商容易积累多头亏损

因此 reservation price 修正为:

\[
r_t
=
S_t 
- q_t \gamma \sigma^2 (T-t)
- q_t f_t S_t (T-t).
\]

如果 funding 高达年化 100–300\%,这项影响极大。

\section{跨交易所做市(Multi-Venue Market Making)}

做市商通常同时挂盘于:

- Binance Futures
- OKX Futures
- Bybit Futures
- Binance Spot
- Coinbase Spot

此时需要建模基差:

\[
B_t = S_t^{(1)} - S_t^{(2)}.
\]

基差具有极强均值回复(Makarov--Schoar 2021),因此做市商可利用:

- 多交易所 inventory balancing  
- 自然对冲(hedging across venues)  
- 更低的整体库存风险  

GLT 模型扩展为向量:

\[
\mathbf{q}_t = (q_t^{(1)}, q_t^{(2)},\dots)
\]

价值函数变为高维 ODE:

\[
\dot{v}_{\mathbf{q}}(t)
=
A(\mathbf{q}) v_{\mathbf{q}}
-
\sum_{i} \eta_i v_{\mathbf{q} + e_i}
-
\sum_{i} \eta_i v_{\mathbf{q} - e_i}.
\]

工程上通常采用低维投影(如 principal hedge factor)。

\section{加入 microprice / imbalance / queue 信息的信号驱动做市}

做市不应是完全被动的。

Stoikov(2018)提出 microprice:

\[
\text{microprice}_t =
\frac{a_t V_t^{bid} + b_t V_t^{ask}}{V_t^{bid} + V_t^{ask}}.
\]

若:

\[
\text{microprice}_t > m_t,
\]

价格有上涨预期,应适当:

- 撤销 ask 单;
- 缩小 bid–ask 方向性倾斜;
- 加速去库存;

可将其并入 AS 的 reservation price:

\[
r_t^{new}
=
r_t + \alpha (\text{microprice}_t - m_t).
\]

其中 $\alpha$ 是信号灵敏度。

此外,可以加入:

- queue size(排队深度)
- imbalance(订单不平衡度)
- Markov-switching volatility regime
- short-term alpha signals(如自回归、microstructure alpha)

这形成 “predictive market making”。

\section{本章小结}

本章构建了一个完整的扩展做市模型族,使得 AS/GLT 在真实 crypto CEX 微观结构中具有可行性。扩展包括:

\begin{itemize}
  \item Hawkes 自激订单流  
  \item 价格冲击与 adverse selection  
  \item maker rebate / taker fee  
  \item 跳跃风险  
  \item 永续合约 funding rate  
  \item 多交易所库存控制  
  \item 基于 microprice/imbalance 的 alpha 驱动做市  
\end{itemize}

这些扩展将在下一章中的校准与估计中发挥核心作用,为第 9–11 章的实盘系统实现提供数学基础。