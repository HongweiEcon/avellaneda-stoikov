\chapter{参数估计与做市模型的校准}
\label{chap:calibration}

\section{引言}

在前几章中,我们构建了做市商面临的数学框架,包括价格模型、订单流模型、库存动态以及最优控制方程。然而,要使做市策略在真实市场中运行,必须估计模型中的关键参数。

本章系统讨论以下参数的估计方法:

\begin{itemize}
  \item 价格波动率 $\sigma$
  \item 订单到达强度参数 $(A, k)$
  \item Hawkes 自激性订单流参数 $(\mu, \alpha, \beta)$
  \item 跳跃强度与冲击分布 $(\lambda_J, \kappa)$
  \item 永续合约的 funding drift $\mu_t$
  \item 市场冲击核 $G(t)$ 与瞬时冲击 $\eta$
\end{itemize}

同时,我们讨论如何将这些参数集成到做市系统,使其在高频数据流下实现实时更新。

\section{价格波动率 $\sigma$ 的高频估计}

\subsection{基于 mid-price 的 realized volatility}

Crypto CEX 的撮合频率极高,可以用成交价或 mid-price 构建:

\[
\hat{\sigma}_t^2
=
\sum_{i=t-\Delta t}^t (m_i - m_{i-1})^2.
\]

常用窗口:

- 高频做市:1–5 秒  
- 中频做市:30–300 秒  

为稳定估计,可使用 EWMA:

\[
\hat{\sigma}_t^2
=
\lambda \hat{\sigma}_{t-1}^2
+
(1-\lambda)(m_t - m_{t-1})^2.
\]

其中 $\lambda \approx 0.9$–$0.99$。

\subsection{带跳跃的波动率估计}

如果价格存在跳跃,应使用:

\[
\hat{\sigma}_t^2 = RV_t - \sum_{j \in \text{jumps}} (\Delta S_j)^2.
\]

跳跃可通过 bipower variation 判定:

\[
BV_t = \frac{\pi}{2}\sum |r_{i-1}||r_i|.
\]

当

\[
|r_i|^2 - BV_t > 3 \cdot \text{std}(BV)
\]

则视为跳跃。

\section{订单流强度参数 $(A, k)$ 的估计}

AS/GLT 强度模型:

\[
\lambda(\delta) = A e^{-k\delta}.
\]

对于成交事件 \((\delta_i,N_i)\),MLE 估计为:

\[
\ln L(A,k)
=
\sum_i \ln\left(Ae^{-k\delta_i}\right)
-
A\sum_i \int e^{-k\delta_i(t)} \mathrm{d}t.
\]

求导可得:

\[
\hat{k}
=
\operatorname{argmin}_k
\left(
\sum_i k\delta_i + A(k) T
\right).
\]

实际中更简单方法:

\[
\ln \lambda_i = \ln A - k\delta_i.
\]

拟合线性模型:

\[
y_i = \alpha + \beta x_i,
\quad
y_i=\ln \lambda_i,\, x_i=\delta_i.
\]

即可得:

\[
k = -\beta,\qquad A = e^\alpha.
\]

Crypto CEX 的经验:

- $k$ 可数秒内变化(市场紧张时变大)
- $A$ 在流动性事件中跳变(如清算)

因此应使用 rolling regression。

\section{Hawkes 自激性订单流参数 $(\mu,\alpha,\beta)$}

对买/卖方向分别估计:

\[
\lambda_t = \mu + \sum_{t_i<t} \alpha e^{-\beta(t-t_i)}.
\]

log-likelihood:

\[
\ln L
=
\sum_{i}
\ln \lambda_{t_i}
-
\int_0^T \lambda_t\, \mathrm{d}t.
\]

对应:

\[
\int_0^T \lambda_t\,\mathrm{d}t
=
\mu T
+
\sum_{t_i<t_j} \frac{\alpha}{\beta}
\left(
1-e^{-\beta(t_j-t_i)}
\right).
\]

可通过数值优化求解:

- BFGS
- L-BFGS-B
- Newton-Raphson

参数含义:

- $\alpha/\beta$ 越接近 1 → 订单流越有簇集 → 做市风险越大
- Crypto 中常见 $\alpha/\beta \in [0.6,0.9]$

可实时更新(每 1–5 秒)。

\section{市场冲击参数的估计}

采用 Gatheral (2010) 的冲击模型:

\[
\Delta S_t = \eta v_t + \sum G(u)v_{t-u}\mathrm du + \epsilon_t.
\]

估计方法:

- 对市场订单大小与即时价格变化回归
- 对 transient impact 进行核回归或 GMM

例如瞬时冲击:

\[
\Delta S_t = \eta q_t + \epsilon_t.
\]

用最小二乘估计:

\[
\hat{\eta}
=
\frac{\sum q_t \Delta S_t}{\sum q_t^2}.
\]

\section{跳跃参数 $(\lambda_J,\kappa)$ 的估计}

跳跃模型:

\[
\mathrm{d}S_t = \sigma \mathrm{d}W_t + \kappa \mathrm{d}J_t.
\]

跳跃检测可用:

- bipower variation
- threshold method

跳跃幅度 \(\kappa\) 用跳跃样本平均:

\[
\hat{\kappa} = \text{mean}(|\Delta S_{jump}|).
\]

跳跃强度:

\[
\hat{\lambda}_J = \frac{\text{jump count}}{T}.
\]

Crypto 市场跳跃频繁,因此模型应动态调整。

\section{永续 funding drift 的估计}

永续价格满足:

\[
\mathrm{d}S_t = f_t S_t \mathrm{d}t + \sigma\mathrm{d}W_t.
\]

实时 drift:

\[
\mu_t = f_t S_t.
\]

但 funding 通常 8 小时结算一次,因此短期影响可通过:

\[
\mu_t^{eff}
=
\mathbb{E}[f_{next}] \cdot S_t
\]

估计未来 funding 可用:

- Net basis
- Premium Index
- Orderbook imbalance
- Funding prediction ML model

\section{多交易所基差与相关性参数的估计}

若做市跨多个交易所:

\[
B_t = S_t^{(1)} - S_t^{(2)}.
\]

基差均值回复模型:

\[
\mathrm{d}B_t = -\kappa_B B_t \mathrm{d}t + \sigma_B \mathrm{d}W_t.
\]

估计方法:

- OLS 回归(equilibrium relationship)
- 2SLS 或 ECM(error correction model)

相关性矩阵:

\[
\Sigma_{ij} = \text{Corr}(\mathrm{d}S_t^{(i)}, \mathrm{d}S_t^{(j)}).
\]

用于库存聚合管理(GLT 扩展)。

\section{模型参数的实时更新机制}

Crypto 做市系统一般每 1 秒或每 5 秒刷新全部参数:

\begin{itemize}
  \item $\sigma$:1–5 秒 realized volatility
  \item $A,k$:rolling regression (过去 50–200 成交)
  \item Hawkes $(\mu,\alpha,\beta)$:moving MLE(30–60 秒窗口)
  \item funding drift:来自交易所 API,插值得到短期 drift
  \item impact $\eta$:以 1 分钟窗口回归更新
  \item jump intensity:rolling jump detection
\end{itemize}

参数刷新后:

- 更新 AS/GLT 最优报价;
- 更新库存 penalty;
- 更新风险限制;
- 重新生成 bid/ask 挂单。

这部分将在第 9–11 章的工程系统中完整实现。

\section{本章小结}

本章提供了做市模型校准的完整方法,包括:

\begin{itemize}
  \item 波动率估计(realized volatility / EWMA / jump-adjusted)
  \item 强度参数 $(A,k)$ 的 MLE 与回归估计
  \item Hawkes 参数的 log-likelihood 估计
  \item 冲击核、跳跃参数、funding drift 的估计方法
  \item 多交易所相关结构的估计
  \item 高频 trading 系统的实时参数更新方案
\end{itemize}

这些参数是将理论做市模型落地为可运行交易系统的关键。

下一章将进入实现层面:如何把这些模型嵌入真实高频做市系统。