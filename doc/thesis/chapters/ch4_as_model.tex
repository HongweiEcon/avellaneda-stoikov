\chapter{Avellaneda--Stoikov 最优做市模型}
\label{chap:as_model}

\section{引言}

Avellaneda--Stoikov(2008)模型是现代自动做市理论的基石,为做市商在随机价格与随机订单流环境下的最优 bid/ask 报价提供了一个可求解、结构清晰、经济含义强烈的动态模型。本章将从基本假设出发,结合上一章的数学基础,系统地推导最优报价、最优库存管理机制以及 reservation price(保留价格)与 optimal spread(最优价差)的闭式解,并解释其在加密货币 CEX 市场中的意义。

本章结构如下:

\begin{itemize}
  \item 第 \ref{sec:as_setup} 节介绍模型设定;
  \item 第 \ref{sec:as_intensity} 节介绍订单到达强度与流动性参数;
  \item 第 \ref{sec:as_hjb} 节推导核心 HJB 方程;
  \item 第 \ref{sec:as_solution} 节给出最优 bid/ask 的显式解;
  \item 第 \ref{sec:as_econ} 节解释解的经济含义;
  \item 第 \ref{sec:as_crypto} 节讨论在加密货币 CEX 环境中的适配;
\end{itemize}

本章提供了做市商与价格风险、库存风险之间权衡的数学基础,是后续带库存约束的 Gueant--Lehalle--Tapia 模型(第五章)的起点。

\section{模型设定}
\label{sec:as_setup}

做市商面对以下状态变量:

\begin{itemize}
  \item $S_t$: 中间价(mid-price)
  \item $q_t$: 库存(inventory)
  \item $X_t$: 现金账户(cash account)
\end{itemize}

总资产(wealth)为:
\[
W_t = X_t + q_t S_t.
\]

中间价遵循布朗运动:
\begin{equation}
\mathrm{d}S_t = \sigma\, \mathrm{d}W_t,
\end{equation}
其中 $\sigma$ 为瞬时波动率,$W_t$ 为标准布朗运动。

做市商在时间 $t$ 处提交两个限价单:

\[
S_t^b = S_t - \delta_t^b,\qquad 
S_t^a = S_t + \delta_t^a,
\]

其中:

- $\delta_t^b$:bid 偏离 mid-price 的距离;
- $\delta_t^a$:ask 偏离 mid-price 的距离;

做市商的决策变量是 $(\delta_t^b, \delta_t^a)$。

库存与现金的动态:

\[
q_t \to q_t + 1 \quad \text{(bid 成交)},
\quad 
X_t \to X_t - (S_t - \delta_t^b),
\]

\[
q_t \to q_t - 1 \quad \text{(ask 成交)},
\quad 
X_t \to X_t + (S_t + \delta_t^a).
\]

做市商希望最大化最终效用:

\[
\max_{\delta^b,\delta^a}
\mathbb{E}\left[ -e^{-\gamma (X_T + q_T S_T)} \right].
\]

其中 $\gamma > 0$ 是绝对风险厌恶度。

\section{订单到达强度模型}
\label{sec:as_intensity}

关键假设:成交强度随报价偏离 mid-price 的距离指数衰减(源自论文 \cite{AvellanedaStoikov2008}):

\begin{equation}
\lambda^b(\delta) = A e^{-k\delta},\qquad
\lambda^a(\delta) = A e^{-k\delta},
\label{eq:intensity}
\end{equation}

其中:
\begin{itemize}
  \item $A$:流动性规模;
  \item $k$:订单对价格敏感度;
\end{itemize}

经济直觉:

- $\delta$ 越小(接近 mid-price) → 成交概率高;
- $\delta$ 越大 → 成交概率低;
- $A,k$ 共同刻画 LOB 的深度与流动性条件。

后续 crypto 扩展模型会引入 bid/ask 非对称强度。

\section{HJB 方程推导}
\label{sec:as_hjb}

定义价值函数:

\[
u(t,x,q,s)=
\max_{\delta^b,\delta^a}
\mathbb{E}_t\left[ -e^{-\gamma(X_T + q_T S_T)} \right].
\]

使用动态规划原理与 Itô 引理,得 HJB 方程:

\[
0 = 
\partial_t u + \frac{1}{2}\sigma^2 \partial_{ss}u
+ 
\max_{\delta^b} 
\lambda^b(\delta^b)
\big[
u(t,x-(s-\delta^b), q+1,s) - u(t,x,q,s)
\big]
\]

\[
+
\max_{\delta^a} 
\lambda^a(\delta^a)
\big[
u(t,x+(s+\delta^a), q-1,s) - u(t,x,q,s)
\big].
\]

为求解此 PDE,我们利用 CARA 效用的指数结构:

\[
u(t,x,q,s) = -e^{-\gamma(x+qs)} \phi_q(t,s).
\]

代入,可消去 $x,s$ 项,并使问题只依赖 $q$。

进一步简化得到:

\[
\phi_q'(t) = -\frac{kA}{k+\gamma}
\left(
\phi_{q+1}(t) e^{-\gamma \delta_t^b}
+\phi_{q-1}(t) e^{-\gamma \delta_t^a}
\right).
\]

优化问题化为对 $\delta_t^b,\delta_t^a$ 的一阶条件求解。

\section{最优报价的解析解}
\label{sec:as_solution}

\subsection{reservation price}

令
\[
r_t = S_t - q_t \gamma \sigma^2 (T - t).
\]

这里 $r_t$ 是做市商在库存 $q_t$ 情况下的风险中性价格(indifference price)。

其含义:

- $q_t > 0$(多头) → reservation price 低于 mid-price → 倾向卖出;
- $q_t < 0$(空头) → 倾向买入;
- 库存越大,偏移越强。

\subsection{optimal spread}

AS 推导得最优价差:

\[
\delta_t^b + \delta_t^a
=
\gamma \sigma^2 (T - t)
+
\frac{2}{\gamma}
\ln\left(1 + \frac{\gamma}{k}\right).
\]

其中:

- 第一项是“风险价差”:库存风险越高,spread 越宽;
- 第二项是“流动性价差”:订单流越敏感($k$ 大),spread 越小。

\subsection{最优 bid/ask 价格}

最终得到最优 bid 与 ask:

\[
S_t^{b*}
=
r_t - \frac{1}{2}
\left(
\gamma\sigma^2 (T - t)
+
\frac{2}{\gamma}\ln\left(1+\frac{\gamma}{k}\right)
\right),
\]

\[
S_t^{a*}
=
r_t + \frac{1}{2}
\left(
\gamma\sigma^2 (T - t)
+
\frac{2}{\gamma}\ln\left(1+\frac{\gamma}{k}\right)
\right).
\]

这就是 Avellaneda--Stoikov 最经典的闭式解。

\section{经济含义与结构分析}
\label{sec:as_econ}

解的结构揭示了做市的三大经济力量:

\subsection{1. 库存风险机制}

reservation price:
\[
r_t = S_t - q_t \gamma \sigma^2 (T - t)
\]

是线性库存调整:

- 库存为正 → 降低 bid、抬高 ask;
- 库存为负 → 相反。

这是库存风险管理的数学主导机制。

\subsection{2. 价差分解}

最优价差可写为:

\[
\text{optimal spread}
=
\text{risk spread}
+
\text{liquidity spread}.
\]

解释:

- 风险越大 → 价差越宽;
- 流动性越高($k$越大)→ 最优价差越窄。

\subsection{3. 风险参数的意义}

- $\gamma$ 越大 → 做市商越保守;
- $\sigma$ 越大 → 做市越不安全,需扩大 spread;
- $k$ 越大 → 市场流动性更好,做市更容易。

\section{对加密货币 CEX 的适配}
\label{sec:as_crypto}

尽管 AS 模型起源于股票市场,但对 crypto CEX 的做市仍具有极强解释力。然而必须进行以下适配:
\begin{enumerate}
  \item 加入 drift(funding rate)\\
永续合约价格含有趋势项:  
funding rate $f_t$ 会使价格产生结构性 drift。

可扩展为:

\[
\mathrm{d}S_t = \mu_t \mathrm{d}t + \sigma \mathrm{d}W_t,
\quad
\mu_t = f_t \cdot S_t.
\]

\item 加入 bid/ask 非对称订单流\\
crypto CEX 常呈现非对称订单流:

\[
\lambda_t^b = A_b e^{-k_b \delta^b},\quad 
\lambda_t^a = A_a e^{-k_a \delta^a}.
\]

例如大户买盘推动、清算瀑布导致一边订单激增。

\item 加入费用结构(maker rebate)\\
maker rebate 会导致 “有效价差” 变化:

\[
\delta^{b,eff} = \delta^b - \text{rebate}.
\]

\item 加入跳跃与高波动区间报价\\
高波动期间应快速扩大 spread,可以把 $\sigma$ 换为实时估计的 $\hat{\sigma}_t$:

\[
\hat{\sigma}_t^2 = \text{realized volatility over past 1–5 seconds}.
\]
\end{enumerate}
\section{本章小结}

本章从最基本的市场设定出发,通过随机控制与指数效用框架,推导了 Avellaneda--Stoikov 最优做市模型的经典解析解,包括 reservation price、optimal spread 与动态报价机制。这一模型提供了做市商在库存风险、价格风险与订单到达概率之间权衡的数学基础。

下一章将基于此模型,引入库存上限与线性化技巧,推导现代工程上最可实现的 Gueant--Lehalle--Tapia 模型。