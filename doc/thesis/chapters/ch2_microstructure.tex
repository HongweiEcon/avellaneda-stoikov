\chapter{限价订单簿与加密货币 CEX 的市场微观结构}
\label{chap:microstructure}

\section{引言}

限价订单簿(Limit Order Book, LOB)是现代电子交易市场中价格形成的核心机制,负责记录市场参与者在不同时点对不同价格的买卖意愿。对于做市商而言,理解订单簿的动态结构、订单的到达与撤销规律、价差与深度的变化机制,是构建精确报价策略与库存管理方法的基础。

本章旨在从数学与经济学的角度,系统介绍 CEX(centralized exchanges)中 LOB 的定义、结构、动态,以及加密货币市场中特有的微观结构特征。内容将涵盖订单类型、队列优先级、订单流的随机结构、价格形成机制、微价格(microprice)、订单簿不平衡度(imbalance)、市场冲击与订单簿弹性等关键概念。我们将结合传统市场微观结构理论\cite{GlostenMilgrom1985,HoStoll1981,Kyle1985,Hasbrouck2007Book,BouchaudBook2010}以及针对加密货币市场的最新研究成果\cite{MakarovSchoar2021,DonierBouchaud2015Crypto},构建一个完整的 LOB 分析框架。

\section{限价订单簿的基本结构}

\subsection{订单簿的形式化定义}

一个标准的双边限价订单簿可以表示为一个时间依赖的价格–数量函数:
\[
\mathcal{L}_t = \{ (p, q(p,t)) : p \in \mathbb{R} \},
\]
其中 $q(p,t)$ 表示价格为 $p$ 时的挂单数量:

- $q(p,t) > 0$ 对应卖单量(ask side)
- $q(p,t) < 0$ 对应买单量(bid side)

在实际交易所中,价格是离散的(tick size $\Delta p$),因此订单簿由价格阶梯(price levels)构成:

\[
\mathcal{L}_t = \{(p_i, q_i(t)), i \in \mathbb{Z}\}.
\]

关键价格概念包括:

\[
\text{best bid: } b_t = \max\{p_i : q_i(t) < 0\},
\]
\[
\text{best ask: } a_t = \min\{p_i : q_i(t) > 0\},
\]
\[
\text{mid-price: } m_t = \frac{a_t + b_t}{2}.
\]

订单簿的即时卖买差(spread)为:
\[
s_t = a_t - b_t.
\]

CEX 市场的 tick size 通常固定,如 Binance 的 BTCUSDT tick size 为 0.1 USDT,而一些期货与 CEX perpetual 合约的 tick size 更小。

\subsection{订单类型与撮合机制}

交易所中存在三类基本订单:

\begin{itemize}
  \item \textbf{限价单(limit order)}:给出价格与数量,进入订单簿排队。
  \item \textbf{市价单(market order)}:立即与已有挂单撮合。
  \item \textbf{撤单(cancellation)}:从订单簿中移除。
\end{itemize}

订单簿的核心机制是 \textbf{价格优先、时间优先(price–time priority)}。对于做市商而言,这意味着:

- 提前挂单有更好的队列位置;
- 小幅提高或降低报价可显著提高成交概率;
- 每一次 cancel/replace 都会丢失队列位置,从而影响预期收益。

\subsection{撮合引擎与队列动态}

限价订单簿可以用三个主要的点过程描述:

- 限价单到达过程 $N^{LO}_t$,带有价格与数量属性;
- 市价单(或 crossing limit order)的到达过程 $N^{MO}_t$;
- 撤单到达过程 $N^{C}_t$。

Cont、Stoikov 与 Talreja\cite{ContStoikovTalreja2010} 提出了一个经典模型,将订单簿视为由一系列独立的 Poisson process 驱动的高维马尔可夫过程,能够解释:

- spread 的稳态分布;
- 深度分布的稳定性形态;
- 订单流不均匀导致的价格跳动。

在加密货币 CEX 中,上述过程更适合使用 Hawkes 过程\cite{BacryMuzyReview2015}建模,因为订单流具有显著的自激性,即大单消耗会引发更多同方向订单到达。

\section{价格形成机制}

\subsection{微价格(microprice)}

简单的 mid-price 不能完全反映短期价格预期,特别在订单簿不平衡时。Stoikov(2018)\cite{Stoikov2018Microprice} 提出“微价格”概念,用订单簿加权反映短期预期:

\[
\text{microprice}_t =
\frac{a_t \cdot V^{bid}_t + b_t \cdot V^{ask}_t}{V^{bid}_t + V^{ask}_t},
\]
其中 $V^{bid}_t$ 和 $V^{ask}_t$ 分别是买卖一档的挂单量。

若买方挂单更多,则 microprice 较接近 ask,表明短期上涨概率更高。

做市商可利用 microprice 调整 reservation price,从而提高 inventory 的动态控制效果。

\subsection{订单簿不平衡度(imbalance)}

经典定义为:

\[
I_t = \frac{V^{bid}_t - V^{ask}_t}{V^{bid}_t + V^{ask}_t}.
\]

实证研究(如 Huang–Stoll\cite{HuangStoll1994})显示,对高频价格变化具有显著预测力。

在加密 CEX 中,imbalance 的预测能力更强,因为:

- maker rebate 激励导致挂单量更可控;
- 大量高频参与者同步观察同一指标。

\section{订单流的统计特性}

\subsection{Poisson 与非齐次 Poisson 模型}

最基础的假设(AS 模型使用)是:

\[
\lambda(\delta) = A e^{-k\delta},
\]

反映报价远离 mid-price 时成交概率下降。

但是这一模型无法解释:

- 市价单爆发(bursts)
- order flow clustering
- volatility–volume coupling

\subsection{Hawkes 过程}

加密货币 CEX 中订单流呈现典型自激性。Bacry–Muzy\cite{BacryMuzyReview2015}模型定义如下:

\[
\lambda_t = \mu + \sum_{t_i < t} \phi(t-t_i),
\]

常用的指数核:

\[
\phi(t) = \alpha e^{-\beta t}.
\]

当 $\alpha/\beta \to 1$ 时,市场高度不稳定,出现 liquidity crisis 或流动性枯竭。

做市商的风险管理必须显式考虑这一点,因为:

- 强自激性意味着库存风险在高频尺度被放大;
- cancellation clustering 意味着队列位置变动频繁;
- burst 模式导致 spread 应在短期扩大。

\section{市场冲击与价格弹性}

市场冲击(market impact)描述市价单对未来价格的影响。Gatheral(2010)\cite{Gatheral2010Impact}指出,冲击应分为:

- permanent impact  
- transient impact  
- instantaneous impact  

冲击大小决定:

- maker 的 adverse selection 风险  
- 被动成交后价格向不利方向移动的概率  

做市商在报价时必须平衡:

- narrow spread → 高成交但高冲击风险  
- wide spread → 低成交但低风险  

这正是 AS-HJB 模型中最关键的 trade-off。

\section{加密货币 CEX 的微观结构特征}

\subsection{永续合约与 funding rate}

Makarov–Schoar(2021)\cite{MakarovSchoar2021}指出:

- perp price = spot price + basis  
- funding rate 驱动 perp price drift  

做市商在 perp 合约上必须将 drift 纳入 reservation price:

\[
r_t = m_t - q_t \gamma \sigma^2 + \text{funding\_adjustment}.
\]

\subsection{跨交易所价格发现与套利约束}

加密市场存在:

- 不同交易所的撮合延迟差异  
- 资金费率差异  
- 资金成本与风控规则差异  

做市商的库存管理必须考虑 cross-venue spread 和基差风险。

\subsection{延迟、架构与 queue-jump 风险}

CEX 市场没有统一撮合系统,不同交易所:

- 服务器架设位置不同  
- 撮合延迟不同  
- API 限速与播单(order stream)结构不同  

延迟越高,queue-jump 越严重,AS 模型中的到达强度需要进行 latency-aware 修正。

\section{本章小结}

本章从形式化定义、订单流、价格形成、市场冲击到加密货币 CEX 的特殊性,系统讨论了限价订单簿的微观结构。通过将传统微观结构理论与加密市场的独特机制相结合,我们为后续的做市模型、库存控制与工程系统搭建了语义与结构基础。

下一章将深入介绍做市问题的数学基础,包括点过程模型、Hawkes 过程、随机微分方程与 HJB 方程,为 Avellaneda–Stoikov 模型与后续扩展提供严谨的数学框架。