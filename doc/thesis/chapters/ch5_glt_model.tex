\chapter{Gueant--Lehalle--Tapia 模型:库存约束、线性化与谱分析}
\label{chap:glt_model}

\section{引言}

上一章给出了 Avellaneda--Stoikov(AS)最优做市模型的完整推导,并得到了 reservation price 与 optimal spread 的经典闭式解。然而,AS 模型存在以下两个局限:

\begin{enumerate}
  \item \textbf{库存无限($q$ 可任意大)}:实际交易中做市商受到严格的风险上限,例如 $|q| \le Q$;
  \item \textbf{HJB 非线性}:AS 的 HJB 方程在加入库存约束后变得更复杂,不易数值求解,更不易在高频系统中实时运行;
\end{enumerate}

为解决这些问题,Guéant、Lehalle 与 Tapia(GLT)在 2012 年的论文\cite{GueantLehalleTapia2012} 提出了一套极为重要的模型扩展,主要结果包括:

\begin{itemize}
  \item 引入库存上限(finite inventory buffer zone);
  \item 通过变量变换将非线性 HJB \emph{线性化};
  \item 建立关于库存 $q$ 的有限维线性 ODE 系统;
  \item 使用谱分解方法(spectral decomposition)求解;
  \item 基于极限情形给出 asymptotic closed-form solution;
\end{itemize}

GLT 模型是目前在实盘做市系统中最广泛使用的数学模型,具备可解释性、可实现性以及良好的数值稳定性。

本章将从 AS 模型出发,完整推导 GLT 模型的核心结果。

\section{有限库存约束与价值函数变换}

回顾 AS 模型的价值函数形式:

\[
u(t,x,q,s)= -e^{-\gamma(x+qs)} \phi_q(t).
\]

GLT 的关键 insight 是:当 $q$ 取有限值集合 $\{-Q,-Q+1,\dots,Q\}$ 时,可以将效用函数完全分离为:

\[
u(t, x, q, s)
=
-e^{-\gamma(x + qs)} 
v_q(t)^{-\gamma/k},
\quad q \in \{-Q,\dots, Q\}.
\]

其中:

- $v_q(t)$ 是新的未知函数;
- $k$ 是订单流敏感度(来自强度函数 $\lambda = A e^{-k\delta}$);
- $Q$ 是库存上限。

代入 HJB 后,可将非线性 PDE 消去,并得到关于 $v_q$ 的线性方程。

\section{线性 ODE 系统的推导}

代入 AS 强度:

\[
\lambda(\delta) = A e^{-k\delta},
\]

并对 $\delta_t^{b,a}$ 求最优解,可以得到最优价差只依赖于 $v_q$ 的比值:

\[
\delta^{b*}(q) 
= 
\frac{1}{k} 
\ln\left(\frac{v_q}{v_{q+1}}\right)
+
\frac{1}{\gamma}\ln\left(1+\frac{\gamma}{k}\right),
\]

\[
\delta^{a*}(q) 
= 
\frac{1}{k} 
\ln\left(\frac{v_q}{v_{q-1}}\right)
+
\frac{1}{\gamma}\ln\left(1+\frac{\gamma}{k}\right).
\]

将这些代回 HJB,即得:

\[
\dot{v}_q(t)
=
\alpha q^2 v_q(t)
-\eta 
\big(v_{q-1}(t) + v_{q+1}(t)\big),
\quad |q| < Q,
\]

边界条件:

\[
\dot{v}_Q = \alpha Q^2 v_Q - \eta v_{Q-1},
\quad
\dot{v}_{-Q} = \alpha Q^2 v_{-Q} - \eta v_{-Q+1},
\]

其中常数:

\[
\alpha = \frac{k}{2\gamma}\sigma^2,
\quad
\eta = A(1+\frac{\gamma}{k})^{-(1+k/\gamma)}.
\]

这样,原先高维、非线性、含位置、含现金的 HJB 问题,变成一个关于库存维度的 \textbf{$(2Q+1)$-维线性常微分方程组}:

\[
\dot{\mathbf{v}}(t) = M \mathbf{v}(t),
\quad
\mathbf{v}(T) = (1,1,\dots,1).
\]

其中矩阵 $M$ 为三对角矩阵:

\[
M_{q,q} = \alpha q^2,
\quad 
M_{q,q+1} = M_{q,q-1} = -\eta.
\]

这就是 GLT 模型的核心。

\section{谱分解求解}

矩阵 $M$ 是一个带状对称矩阵:

\[
M =
\begin{pmatrix}
\ddots & -\eta & 0 \\
-\eta & \alpha q^2 & -\eta \\
0 & -\eta & \ddots
\end{pmatrix}.
\]

由于 $M$ 是对称的,它拥有一组标准正交特征向量 $\{f^i\}$ 以及对应特征值 $\{\lambda_i\}$:

\[
M f^i = \lambda_i f^i.
\]

因此线性系统可解为:

\[
v_q(t)
=
\sum_{i=0}^{2Q}
c_i f^i_q e^{\lambda_i (t-T)},
\]

其中 $c_i$ 由终端条件 $v_q(T)=1$ 决定。

在实际实现中,$Q$ 通常不大(5–50),求解成本非常低。

\section{渐近闭式解(Asymptotic Closed Form)}

当 $Q$ 很大、$\eta/\alpha$ 处于适中规模时,可以证明最小特征值对应的特征向量渐近为:

\[
f^0_q \approx C e^{-\omega q^2},
\quad
\omega = \frac{1}{2}\sqrt{\frac{\alpha}{\eta}}.
\]

代回 bid/ask 价差公式,可得 asymptotic closed form:

\[
\delta^{b*}(q) 
\approx
\frac{1}{\gamma}
\ln\left(1+\frac{\gamma}{k}\right)
+
\frac{1}{2k}\sqrt{\frac{\alpha}{\eta}}(2q+1),
\]

\[
\delta^{a*}(q) 
\approx
\frac{1}{\gamma}
\ln\left(1+\frac{\gamma}{k}\right)
-
\frac{1}{2k}\sqrt{\frac{\alpha}{\eta}}(2q-1).
\]

与 AS 模型相比,这一版本:

- retain 完整库存非线性行为;
- 对大库存具有严格惩罚;
- 在高频系统中容易实时计算。

这是目前工程中最常用的做市公式之一。

\section{经济解释}

\subsection{库存越大,报价越偏离 mid-price}

bid/ask 的库存调整项:

\[
\pm \frac{1}{2k}\sqrt{\frac{\alpha}{\eta}}(2q \pm 1)
\]

说明:

- $q>0$(多头) → bid 往下调整、ask 往上调整;
- $q<0$(空头) → 反之。

库存越大,偏移越强。这与 AS 模型一致,但惩罚更加平滑。

\subsection{价差自动扩张}

最优 spread:

\[
\delta^{a*}(q)+\delta^{b*}(q)
=
\frac{2}{\gamma}
\ln\left(1+\frac{\gamma}{k}\right)
+
\sqrt{\frac{\alpha}{\eta}}.
\]

价差自动包含:

- 流动性惩罚 $\frac{2}{\gamma}\ln(1+\gamma/k)$;
- 库存惩罚 $\sqrt{\alpha/\eta}$。

\section{在加密货币 CEX 做市的适配}

GLT 模型在 crypto CEX 中比在传统股票市场更有价值:

\begin{itemize}
  \item 高波动性 $\sigma$ → 增强库存风险控制需求;
  \item 订单簿深度不稳定 → $A,k$ 经常要实时估计;
  \item 自激型订单流 → $\eta$ 可能动态变化;
  \item 多交易所报价 → $Q$ 需要动态调整;
\end{itemize}

在实际 BTC/ETH 做市系统中,通常使用:

- GLT asymptotic closed-form 用于高频更新;
- $A,k,\sigma$ 用 1–10 秒窗口实时估计;
- $Q$ 根据 vol、资金、风险限额动态调整。

\section{本章小结}

本章从 Avellaneda--Stoikov 模型出发,引入库存约束,并通过变量变换与线性化技巧,将做市问题转化为有限维线性 ODE 系统。通过谱分解可以求得解析形式的数值解,而渐近闭式解在工程系统中最为常用。

下一章将基于本章的库存结构,研究做市盈利的本质,并结合均值回复模型与加密货币价格行为分析长期做市收益。