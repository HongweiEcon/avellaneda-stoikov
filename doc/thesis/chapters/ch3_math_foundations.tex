\chapter{做市问题的数学基础}
\label{chap:math_foundations}

\section{引言}

自动做市的本质是一个典型的随机控制问题:在随机演化的价格过程与订单流过程中,做市商通过选择动态 bid/ask 报价与挂单数量,最大化某种风险调整后的目标函数。在这样的框架下,必须运用到随机微分方程(SDE)、点过程理论(Poisson 与 Hawkes 过程)、随机最优控制理论(HJB 方程),并构建市场微观结构层面的数学表述。

本章旨在为 Avellaneda--Stoikov 以及后续章节中的最优做市模型建立严格的数学基础。内容包括:价格过程的随机建模、订单流的点过程建模、效用函数与风险厌恶表示、库存动态的随机结构,以及 HJB 方程的推导与解释。

\section{价格过程的随机建模}

做市问题的第一层结构是价格动态模型。尽管真实市场中的价格表现包含跳跃、冲击与自激性,但在理论分析中,最基础的模型仍是布朗运动:

\[
\mathrm{d}S_t = \sigma \,\mathrm{d}W_t,
\]
其中 $W_t$ 为标准布朗运动。

这一模型具有以下优点:

\begin{itemize}
  \item 价格路径连续,避免了大量数学上的不连续项处理;
  \item $\sigma$ 代表瞬时波动率,是做市模型中最核心的参数;
  \item 与效用最大化框架结合可产生显式最优报价。
\end{itemize}

更一般的模型包括:

\subsection{含漂移项的 SDE}
加密永续合约中,由于 funding rate 引入长期漂移,合理的模型应包含:

\[
\mathrm{d}S_t = \mu_t \,\mathrm{d}t + \sigma_t \,\mathrm{d}W_t,
\]
其中 $\mu_t$ 可由 funding rate 估计(参见 Makarov--Schoar \cite{MakarovSchoar2021})。

\subsection{含跳跃的价格过程}

加密货币经常出现跳跃,可以用跳扩散模型:

\[
\mathrm{d}S_t = \sigma\, \mathrm{d}W_t + \kappa \,\mathrm{d}J_t,
\]

其中 $J_t$ 是泊松跳跃过程。跳跃会显著影响最优报价宽度,因为大跳风险增加做市的库存风险。

本论文后续将在第 \ref{chap:extensions} 章讨论跳扩散与做市的扩展模型。

\section{订单流的点过程模型}

做市商通过被动挂单赚取价差,因此成交的到达时间是核心变量。订单流在数学上通常用点过程(counting process)表示。

\subsection{Poisson 模型}

Avellaneda--Stoikov 模型假设成交是指数型随机过程,其到达强度与报价偏离中间价的距离相关:

\[
\lambda^{bid}(\delta) = A e^{-k \delta^b},\qquad
\lambda^{ask}(\delta) = A e^{-k \delta^a}.
\]

该假设具有以下意义:

\begin{itemize}
  \item 贴近 mid-price 的报价更容易成交;
  \item 参数 $k$ 表示订单流对价格的敏感度;
  \item 参数 $A$ 表示市场流动性规模;
  \item 强度的指数型结构使得最优报价可以显式求解。
\end{itemize}

这是做市模型的核心动力来源:spread 与成交概率之间的权衡。

\subsection{非齐次 Poisson 与扩展模型}

在更一般情形下,订单强度可以依赖价格状态、波动率或时间:

\[
\lambda_t = \lambda(S_t, \sigma_t, t).
\]

例如,在波动突增期,订单流会激增,做市商应扩大报价距离以降低风险。

\subsection{Hawkes 过程}

加密货币订单流具有强烈的自激性,Hawkes 过程是更适合的模型\cite{BacryMuzyReview2015}:

\[
\lambda_t = \mu + \sum_{t_i < t} \alpha e^{-\beta (t - t_i)}.
\]

其中:

- $\alpha/\beta$ 越大,自激性越强;
- 订单密集时成交风险更高;
- 做市商必须动态调整报价,避免过多 inventory 累积。

第 \ref{chap:extensions} 章将专门讨论 Hawkes 驱动的做市模型。

\section{现金与库存的动态}

做市商的状态变量包括现金账户 $X_t$ 与库存 $q_t$。  
当 bid 单成交时:

\[
q_t \to q_t + 1,\qquad
X_t \to X_t - (S_t - \delta_t^b).
\]

当 ask 单成交时:

\[
q_t \to q_t - 1,\qquad
X_t \to X_t + (S_t + \delta_t^a).
\]

因此总资产(wealth):

\[
W_t = X_t + q_t S_t.
\]

做市商的风险主要来自 $q_t S_t$ 的波动,即库存风险。  
库存越大,做市商越容易在价格跳动中亏损,这就是为什么做市模型必须包含库存处罚项(inventory penalty)。

\section{效用函数与风险厌恶}

Avellaneda--Stoikov 使用指数效用(CARA):

\[
U(w) = -e^{-\gamma w},
\]

其特点:

- 绝对风险厌恶度恒定;
- 加法结构使得 HJB 推导简单;
- 导致 reservation price 与 inventory 呈线性关系。

效用最大化目标:

\[
\max_{\delta^b,\delta^a} \mathbb{E}_t\left[
    -e^{-\gamma (X_T + q_T S_T)}
\right].
\]

这里 $\gamma$ 决定做市商的风险偏好:

- $\gamma$ 小 → 激进做市商 → spread 较窄;
- $\gamma$ 大 → 保守做市商 → spread 较宽。

\section{随机控制与 HJB 方程}

最优报价策略是一个连续时间随机控制问题。价值函数定义为:

\[
u(t, x, q, s) =
\max_{\delta^b, \delta^a}
\mathbb{E}\left[ -e^{-\gamma (X_T + q_T S_T)} \mid X_t = x, q_t = q, S_t = s \right].
\]

应用动态规划原理(Bellman optimality principle):

\[
0 = \partial_t u + \frac{1}{2}\sigma^2 \partial_{ss}u
+ \max_{\delta^b} \lambda^b(\delta^b) 
  \big[ u(t, x - (s-\delta^b), q+1, s) - u(t,x,q,s) \big]
\]
\[
+ \max_{\delta^a} \lambda^a(\delta^a) 
  \big[ u(t, x + (s+\delta^a), q-1, s) - u(t,x,q,s) \big].
\]

这是 Avellaneda--Stoikov 模型的核心 HJB。

简化后可导出:

- reservation price  
- optimal spread  

其推导将在下一章详细展开。

\section{库存风险与二次罚项}

另一类做市模型(如 Cartea--Jaimungal)采用对库存添加显式二次罚项:

\[
\text{Objective} = 
\mathbb{E}[W_T] - \frac{\gamma}{2}\mathbb{E}\left[ \int_0^T q_t^2 \sigma_t^2 \mathrm{d}t \right].
\]

意义:

- 直接惩罚大库存;
- 与布朗价格模型完美匹配;
- 适用于非指数效用的框架。

本论文仍采用 AS 模型的 CARA 效用,但库存罚项概念对工程系统非常重要(将在第 \ref{chap:system_design} 章讨论)。

\section{本章小结}

本章介绍了做市问题的数学基础,包括:

- 价格过程:布朗、含漂移、跳扩散;
- 点过程:Poisson 与 Hawkes;
- 现金与库存动态;
- CARA 效用与风险偏好;
- 随机控制与 HJB 方程框架;

这些工具构成后续 Avellaneda--Stoikov 模型(第 \ref{chap:as_model} 章)与 GLT 模型(第 \ref{chap:glt_model} 章)的基础。

下一章将从本章的基础数学出发,完整推导 Avellaneda--Stoikov 最优做市模型。