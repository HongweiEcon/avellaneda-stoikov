\chapter{自动做市系统架构设计}
\label{chap:system_design}

\section{引言}

在前八章中,我们从市场微观结构、随机控制理论、最优做市模型、扩展模型到参数估计,系统构建了完整的自动做市理论框架。然而,数学模型本身并不能直接在真实市场运行。要将其转化为稳定、高性能、低延迟、可风险控制的交易系统,需要严格的工程架构。

本章将设计一个可部署于加密货币 CEX(如 Binance、OKX、Bybit)的高频自动做市系统。系统由以下核心模块组成:

\begin{enumerate}
  \item 实时数据接入模块(Market Data Feed Handler)
  \item 定价引擎(Pricing Engine)
  \item 订单管理系统(Order Management System, OMS)
  \item 库存与风险控制系统(Risk Engine)
  \item 参数更新引擎(Parameter Refresh Engine)
  \item 报价与执行策略调度器(Strategy Scheduler)
  \item 仿真器与回测引擎(Simulator \& Backtester)
  \item 系统监控与日志框架(Monitoring \& Logging)
\end{enumerate}

这些模块必须协同运行,在毫秒级时间尺度下完成全链条计算与决策。

\section{系统总体架构}

图 \ref{fig:system_architecture}(请在最终论文中放图)展示了系统架构的总览,自上游的数据流到下游的订单执行路径清晰可见。

系统可抽象为三个主要层级:

\begin{itemize}
  \item \textbf{数据层(Data Layer)}:接收交易所原始数据(Orderbook、Trades、Funding、Index Price)
  \item \textbf{逻辑层(Logic Layer)}:定价、风险、策略、参数估计
  \item \textbf{执行层(Execution Layer)}:订单生成、撤单、重挂、成交处理
\end{itemize}

每一层都对系统稳定性、延迟和收益具有决定性影响。

\section{实时数据接入模块}

自动做市是一个强依赖市场微观结构的策略,因此“数据延迟”直接决定做市盈利能力。Crypto CEX 数据一般通过 WebSocket 接收:

\[
\text{Data} = \{ \text{Orderbook L2}, \text{Trade Prints}, \text{Ticker}, \text{Funding Rate}, \dots \}.
\]

要求:

\begin{itemize}
  \item L2 orderbook 更新粒度需达到毫秒级;
  \item Trade Prints 用于订单到达强度、Hawkes 参数与 volatility 估计;
  \item Funding/Index 用于永续 drift 修正;
  \item 所有数据需严格按时间戳排序;
\end{itemize}

数据模块输出:

\[
\mathcal{D}_t = \{m_t, b_t, a_t, \Delta m_t, \text{OB}_t, \text{trades}, \text{imbalance}, V_t^{bid},V_t^{ask}\}.
\]

这些输入会进入定价引擎与参数引擎。

\section{定价引擎(Pricing Engine)}

定价引擎是做市策略的数学核心,它使用:

\begin{itemize}
  \item AS/GLT 模型(或扩展模型)
  \item 实时波动率 $\hat{\sigma}_t$
  \item 强度参数 $(\hat{A}_t, \hat{k}_t)$
  \item 订单流 Hawkes 参数 $(\hat{\mu},\hat{\alpha},\hat{\beta})$
  \item 基差、funding drift
  \item microprice、imbalance、queue-length signals
\end{itemize}

\subsection{核心输出}

定价引擎每秒或每 200ms 输出:

\[
\delta_t^{b*}(q_t),\quad \delta_t^{a*}(q_t)
\]

\[
S_t^{b*} = S_t - \delta_t^{b*},\qquad
S_t^{a*} = S_t + \delta_t^{a*}.
\]

这些价格直接进入订单管理系统。

\subsection{GLT 模型在引擎中的实现}

因为 GLT 提供线性 ODE:

\[
\dot{v}(t) = M v(t),
\]

我们可以:

\begin{itemize}
  \item 离线求特征值/特征向量;
  \item 在线用向量外积直接计算 $v_q(t)$;
  \item 实时更新 $(A,k,\sigma)$;
  \item 对库存变化进行即时调整;
\end{itemize}

这使得 GLT 模型可以毫秒级计算。

\section{订单管理系统(OMS)}

OMS 是做市系统的执行核心,负责:

\begin{enumerate}
  \item 下单(Limit)
  \item 撮合响应(Fill / Partial Fill)
  \item 撤单(Cancel)
  \item 改单(Replace)
  \item 订单生命周期管理(Aging)
\end{enumerate}

\subsection{OMS 必须保证的性质}

\begin{itemize}
  \item 幂等性(Idempotency)
  \item 低延迟(sub-ms 内部处理)
  \item 可恢复性(断线自动恢复挂单)
  \item 与交易所 API 一致性(Orderbook Snapshot 同步)
\end{itemize}

\subsection{订单刷新策略}

典型的做市刷新周期:

- 100ms – 500ms 刷新全部报价  
- 20ms – 50ms 检查是否被市场跳过(queue position risk)  
- 若价格偏离 mid-price 超过阈值立即撤单  

高频做市中一个关键策略是:

\[
\textbf{Cancel aggressive, place conservative.}
\]

OMS 的性能直接影响策略的胜率。

\section{库存与风险控制系统(Risk Engine)}

风险引擎的主要目标是:

\[
|q_t| \le Q_t^{max}.
\]

库存上限 \(Q_t^{max}\) 可动态控制:

\[
Q_t^{max} = f(\sigma_t, \text{Hawkes intensity}, \text{liquidity}, \text{vol regime}).
\]

示例:

\[
Q_t^{max} = \frac{K}{\sigma_t},
\]

即波动越大,库存越小。

Risk Engine 的功能包括:

\begin{itemize}
  \item 实时计算库存风险 $q_t \sigma_t$
  \item 使用 GLT 的库存倾斜调整 ask/bid
  \item 监测 extreme fills、清算行情、跳跃风险
  \item 平仓机制(当库存过大时主动吃单)
\end{itemize}

必要时执行主动对冲:

\[
\text{Hedge} = -q_t S_t^{perp}.
\]

\section{参数更新引擎}

从第八章的校准结果可知,所有参数必须滚动估计:

\[
\Theta_t = \{\sigma_t, A_t, k_t, \mu_t,\alpha_t,\beta_t, \eta_t, f_t,\lambda_J\}.
\]

参数更新周期:

- 高频做市:200ms–1s  
- 中频做市:1–10s  

更新方式:

- EWMA(volatility)
- Rolling regression($k$)
- Hawkes MLE(每 1–5 秒)
- Funding drift 通过 exchange APIs

参数更新后,会立即刷新报价。

\section{策略调度器(Strategy Scheduler)}

Scheduler 是整个系统的 “大脑”,负责:

\begin{itemize}
  \item 控制策略执行顺序
  \item 限制过度撤单(anti-cancel-throttle)
  \item 控制挂单量(per-layer max)
  \item 监控 market state
\end{itemize}

典型 workflow:

\begin{enumerate}
  \item 读取最新市场数据  
  \item 更新参数  
  \item 调用 Pricing Engine 生成报价  
  \item 调用 Risk Engine 调整库存倾斜  
  \item 调用 OMS 执行 / 撤单 / 改单  
\end{enumerate}

所有过程必须在 50–200ms 内完成。

\section{高频仿真器(LOB Simulator)}

回测完全无法模拟真实 CEX 成交,因为:

- queue priority
- market order bursts
- orderbook resiliency
- Hawkes effect
- latency 影响成交事件

因此需要构建 LOB Simulator:

\[
\mathcal{S}: \text{Order stream} \to \text{Simulated fills}.
\]

使用以下数据:

- L2 snapshots
- incremental diff feed
- trade prints
- cancellation bursts

仿真器返回:

- 是否成交?
- 成交价格?
- queue position?
- partial fill 还是 full fill?

真实做市回测必须使用 LOB Simulator。

\section{系统监控}

核心监控指标:

\[
\text{PnL}_t,\quad q_t,\quad \text{spread},\quad \sigma_t,\quad \lambda_t^a,\lambda_t^b,
\]

以及:

- OMS 错误率
- API 延迟
- order sent / filled ratio
- cancel rate
- 市场跳跃检测

监控系统必须能在:

- 清算行情
- network outage
- market halt  

情况下安全关闭系统。

\section{系统延迟与硬件架构}

Crypto CEX API 延迟典型值:

- Binance Futures:5–20ms
- OKX Futures:10–25ms
- Coinbase:20–50ms

系统目标:

\[
\text{internal latency} < 5\text{ ms}.
\]

硬件:

- 单机 + 多线程
- Python + C++/Rust mixed
- affinity pinning(固定 CPU 绑定)
- 高频时需要 binlog-based deterministic replay

\section{本章小结}

本章从工程角度构建了一个完整的加密货币做市系统,包括:

\begin{itemize}
  \item 实时数据接入  
  \item GLT/AS 定价引擎  
  \item OMS 订单执行模块  
  \item 风险管理引擎  
  \item 滚动参数更新  
  \item 仿真器与回测系统  
  \item 系统监控与低延迟设计  
\end{itemize}

本章将理论构建与实际交易系统完美结合,为下一章的 整套 Crypto 做市框架(Spot + Perpetual + Multi-Venue) 提供基础。