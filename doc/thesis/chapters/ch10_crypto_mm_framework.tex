\chapter{加密货币多市场(现货–永续–跨交易所)一体化做市框架}
\label{chap:crypto_mm_framework}

\section{引言}

加密货币市场具有高度复杂的微观结构特征:

\begin{itemize}
  \item CEX 现货(spot)与永续合约(perpetual)高度联动;
  \item 永续合约价格通过 funding rate 与现货锚定;
  \item 多交易所(Binance、OKX、Bybit)之间存在持续基差;
  \item 高频波动显著,订单簿深度不断变化;
  \item 市场存在跳跃与清算驱动的 “流动性瀑布”;
\end{itemize}

因此,一个成功的做市系统不应局限于单一市场,而必须构建:

\begin{center}
\textbf{Spot + Perpetual + Multi-Venue 的一体化做市框架}
\end{center}

本章从数学建模、策略逻辑与工程系统三个维度,构建可直接部署的综合做市结构。

\section{BTC/ETH 现货–永续价格结构}

定义:

\begin{itemize}
  \item $S_t^{spot}$:现货价格(Binance、Coinbase 等)
  \item $S_t^{perp}$:永续合约价格(Binance Futures、OKX Futures)
  \item $f_t$:永续 funding rate
  \item $I_t$:指数价格(index price)
\end{itemize}

永续合约理论上与现货锚定:

\[
S_t^{perp} = S_t^{spot} (1 + b_t),
\quad
b_t \approx \int f_t\,\mathrm{d}t.
\]

因此基差 $b_t$ 应满足均值回复:

\[
\mathrm{d}b_t = -\kappa_b b_t\,\mathrm{d}t + \sigma_b\,\mathrm{d}W_t.
\]

该模型由 Makarov–Schoar(2021)实证支持。

\subsection{对做市的启示}

基差均值回复意味着:

\begin{itemize}
  \item perp–spot 做市天然具有低库存风险;
  \item 可跨市场自对冲(spot/perp hedge);
  \item perp 上的 drift ≈ funding;
\end{itemize}

因此做市不应单独建模 spot/perp,而应构建 unified price:

\[
S_t = S_t^{spot},\qquad 
S_t^{perp} = S_t + \tilde{b}_t.
\]

\section{跨市场库存向量与风险结构}

定义库存向量:

\[
\mathbf{q}_t = 
\begin{pmatrix}
q_t^{spot} \\
q_t^{perp(1)} \\
q_t^{perp(2)} \\
\vdots \\
q_t^{perp(n)}
\end{pmatrix},
\quad
q_t^{spot}, q_t^{perp(i)} \in \mathbb{R}.
\]

定义风险资产价格向量:

\[
\mathbf{S}_t =
(S_t^{spot}, S_t^{perp(1)},\dots,S_t^{perp(n)}).
\]

组合价值:

\[
W_t = X_t + \mathbf{q}_t^\top \mathbf{S}_t.
\]

库存风险的方差:

\[
\mathrm{Var}(\mathrm{d}W_t)
=
\mathbf{q}_t^\top \Sigma_t\, \mathbf{q}_t,
\]

其中 $\Sigma_t$ 是 spot–perp–multi-venue 协方差矩阵。

基差均值回复的存在,使得:

\[
\mathrm{Corr}(S^{spot}, S^{perp}) \approx 0.99,
\]

因此风险可以通过 delta-neutral hedge 有效控制:

\[
q_t^{spot} + \sum_i q_t^{perp(i)} \approx 0.
\]

这构成了 multi-market 做市的基础。

\section{GLT 的高维扩展:多市场库存控制}

将 GLT ODE 从一维扩展到多维:

\[
\dot{v}_{\mathbf{q}}(t)
=
-\mathbf{q}^\top \Gamma\, \mathbf{q}\, v_{\mathbf{q}}
+
\sum_{i=1}^n \eta_i 
\big(
v_{\mathbf{q}+e_i}
+
v_{\mathbf{q}-e_i}
\big),
\]

其中:

- $\Gamma$ 是 inventory penalty matrix(由 $\Sigma_t$ 决定)
- $e_i$ 是第 $i$ 市场的单位库存增量
- $\eta_i$ 对应市场 $i$ 的流动性

高维 GLT 模型太大(状态空间指数级),实际中使用:

- **低秩近似(Low-rank)**
- **主对冲方向(Principal Hedge Direction)**
- **总库存近似(Aggregate Inventory Approximation)**

定义 aggregate inventory:

\[
q_t^{agg} = q_t^{spot} + \sum_{i=1}^n q_t^{perp(i)}.
\]

然后将库存风险简化为:

\[
\mathrm{Risk}(q_t^{agg}) \approx 
\gamma \sigma^2 (T-t) (q_t^{agg})^2.
\]

从而重新使用 GLT 的一维公式:

\[
\delta^{b,a*}(q_t^{agg})
=
\text{GLT}(q_t^{agg}, A, k, \sigma).
\]

这大幅简化计算,足以应对实际交易环境。

\section{多市场报价结构}

对 spot 与 perp 生成独立但相关的报价:

\[
S^{spot,b*} = S_t - \delta^{b*}(q_t^{agg}),
\quad
S^{spot,a*} = S_t + \delta^{a*}(q_t^{agg}),
\]

\[
S^{perp,b*} = S_t^{perp} - \delta^{b*}(q_t^{agg}) - \phi b_t,
\quad
S^{perp,a*} = S_t^{perp} + \delta^{a*}(q_t^{agg}) - \phi b_t,
\]

其中 $\phi$ 控制 perp–spot 基差预期修正(hedge ratio)。

此外永续合约还需加入 funding drift:

\[
r_t^{perp} = S_t^{perp} - q_t \gamma \sigma^2 (T-t) - f_t S_t (T-t).
\]

\section{跨交易所一体化做市}

对于多交易所现货与永续:

定义每个交易所的 mid-price:

\[
m_t^{(i)},\quad i=1,\dots,N.
\]

跨市场价差:

\[
\Delta_{ij} = m_t^{(i)} - m_t^{(j)}.
\]

若:

\[
|\Delta_{ij}| > \theta,
\]

则可以:

- 在 $i$ 交易所挂 ask
- 在 $j$ 交易所挂 bid

形成自然对冲。

库存向量变为:

\[
q_t = (q^{spot}_{Bin}, q^{spot}_{OKX}, q^{perp}_{Bin}, q^{perp}_{OKX},\dots).
\]

Risk Engine 控制:

\[
\sum_i q_t^{spot(i)} + \sum_j q_t^{perp(j)} \approx 0.
\]

\section{做市、对冲与套利的一体化调度器}

统一调度器执行:

\begin{enumerate}
  \item 更新市场数据(spot, perp, basis, funding)
  \item 更新参数($\sigma, A, k,$ Hawkes, funding drift)
  \item 计算 unified inventory $q_t^{agg}$
  \item 计算 GLT 基础报价
  \item 加入基差、funding、冲击等修正
  \item 多市场报价生成
  \item OMS 执行
  \item Risk Engine 检查是否需要对冲
\end{enumerate}

特别是跨市场对冲:

\[
\text{hedge size} = -q_t^{agg}.
\]

若 perp 流动性更好,则优先用 perp 对冲;若 spot 较稳定,可以 spot hedge。

\section{示例:BTC/ETH 做市全流程}

在实际交易中,该框架运行如下:

\subsection{步骤 1:数据更新}

- 获取 Binance/OKX/Bybit Perp L2
- 获取 Binance/Coinbase Spot L2
- 计算 mid-price、funding、basis

\subsection{步骤 2:参数更新}

- $\sigma_t$:1 秒 realized vol
- $A_t,k_t$:rolling regression
- Hawkes 参数:1 秒 MLE
- funding drift:exchange API

\subsection{步骤 3:库存与风险}

计算:

\[
q_t^{agg}
=
q_t^{spot}
+
q_t^{perp}.
\]

若:

\[
|q_t^{agg}| > Q_t^{max},
\]

则立即主动对冲。

\subsection{步骤 4:报价生成}

- 基于 GLT 输出基础 $\delta^*$  
- perp 报价加入 funding 修正  
- spot/perp 加入基差修正  

\subsection{步骤 5:订单执行}

通过 OMS:
\begin{itemize}
  \item cancel \& replace
  \item multi-layer quoting
  \item skip-switching(检测不利行情及时撤单)
\end{itemize} 

\section{本章小结}

本章构建了一个可在真实 crypto 市场部署的完整多市场做市框架。其特点包括:

\begin{itemize}
  \item Spot–Perp 联动报价模型  
  \item 基差均值回复与 unified inventory  
  \item 多交易所协同做市  
  \item 多维库存向量与风险控制  
  \item 资金费率驱动的 drift 修正  
  \item 可实时计算的 GLT-based 价格输出  
  \item 一体化调度器管理全链路  
\end{itemize}

下一章将使用仿真器与历史市场数据对该框架进行严格验证。