\chapter{引言}
\label{chap:intro}

\section{研究背景与动机}

过去二十年中,电子化与自动化交易极大地重塑了金融市场的微观结构。以限价订单簿(Limit Order Book, LOB)为核心的连续双边交易机制已经成为股票、期货、外汇等资产的标准交易模式\cite{BouchaudBook2010,ContStoikovTalreja2010}。在这一框架下,所有买卖意愿以限价单的形式在不同价格层级上排队等候成交,成交价格由主动提交的市价单(或价格跨越对手方报价的限价单)与订单簿上的挂单相互撮合而确定。市场参与者既可以作为“流动性提供者”(liquidity provider)通过挂单赚取价差,也可以作为“流动性需求者”(liquidity taker)通过吃单快速执行交易。

自动做市(automated market making)正是以限价订单簿为载体的一类关键策略:做市商通过持续在买一与卖一附近挂出双边报价,动态调节报价位置与挂单数量,为市场提供流动性,并从买卖价差中获取补偿。早在 Ho 和 Stoll 的经典研究\cite{HoStoll1981,HoStoll1983}以及 Glosten--Milgrom 模型\cite{GlostenMilgrom1985}中,做市商的核心问题就已经被形式化为:在面对信息不对称、库存压力与订单到达不确定性的前提下,如何设定买卖报价与库存调整策略,使得长期利润在风险约束下最大化。此后,Kyle 模型\cite{Kyle1985}进一步强调了信息驱动订单流对价格冲击和做市商盈亏的影响。

进入高频交易时代,做市问题的本质并未改变,但交易环境发生了质的飞跃:撮合速度以微秒计,订单流具有强烈的簇集性和自激性\cite{BacryMuzyReview2015},订单簿结构高度动态化,价格波动集中于极短时间尺度。Avellaneda 与 Stoikov 在 2008 年提出的做市模型\cite{AvellanedaStoikov2008}被广泛视为现代最优做市理论的起点:他们在连续时间、随机订单到达的框架下,将做市商的决策问题形式化为一类随机控制问题,并用 HJB(Hamilton--Jacobi--Bellman)方程刻画最优 bid/ask 报价与库存管理策略。随后,Gueant、Lehalle 与 Tapia 等人\cite{GueantLehalleTapia2012}对该模型进行重要扩展,通过引入适当的变量变换,将原本非线性的 HJB 方程转化为线性常微分方程组,使模型在工程上具备可实现性与可校准性。另一方面,Chakraborty 与 Kearns\cite{ChakrabortyKearns2011}从更加抽象的层面分析了做市策略长期盈利的来源,指出做市利润本质上取决于局部价格波动与长期价格趋势之间的权衡:只有当价格过程具有足够的短期波动而长期趋势不过于强烈时,做市才能在风险可控的前提下获得稳定的期望收益。

与此同时,以比特币(BTC)与以太坊(ETH)为代表的加密货币市场在过去十年间迅速扩张,衍生出以永续合约(perpetual swap)为主导的一整套交易生态系统。与传统股票或期货市场相比,加密货币市场具有若干鲜明特征:交易机制高度电子化、7×24 小时连续交易、现货与永续合约紧密联动、跨交易所价格发现过程复杂、清算机制与 funding rate 反馈影响价格路径等\cite{MakarovSchoar2021}。在这种环境中,自动做市不仅要处理传统意义上的库存风险与价格风险,还必须应对更强烈的波动性、更复杂的市场冲击与更显著的技术约束(如撮合延迟、带宽限制、风控指标实时监控等)。

因此,本论文的研究动机可以概括为:\emph{在加密货币限价订单簿这一高度动态且结构复杂的交易环境中,如何基于坚实的数学基础,构建一套理论上严谨、实践中可实现的最优做市与库存管理体系?}

\section{自动做市与限价订单簿的基本问题}

在限价订单簿框架下,做市商面临的基本决策包括:

\begin{itemize}
  \item \textbf{报价决策(quote placement)}:在当前中间价(mid-price)附近,选择多大的买卖价差(spread),以何种价格水平挂出 bid 与 ask?  
  \item \textbf{库存管理(inventory control)}:在订单随机成交的过程中,如何控制库存水平,使其在可接受的风险范围内波动?  
  \item \textbf{订单大小与挂单结构(order sizing \& layering)}:在各个价位挂出多少数量,是否采用多层挂单结构?  
  \item \textbf{订单生命周期管理(lifecycle management)}:多长时间刷新订单?在何种条件下撤单重挂?如何应对瞬时流动性枯竭或波动突增?  
  \item \textbf{风险与约束(risk constraints)}:如何在给定的风险限额、VaR/CVaR 约束与监管/风控指标下运行做市策略?  
\end{itemize}

传统微观结构理论将做市商面临的成本拆解为三部分:信息不对称导致的逆向选择成本(adverse selection cost)、库存成本(inventory cost)以及订单处理成本(order processing cost)\cite{GlostenMilgrom1985,Hasbrouck2007Book}。在高频环境中,这三类成本通过限价订单簿的形状、订单流的到达强度与撤单行为、以及价格冲击函数的形式共同体现出来。例如,Cont 等人\cite{ContStoikovTalreja2010}提出了一种随机 LOB 模型,将限价单、撤单与市价单到达均表示为点过程,从而可以在微观层面分析价差、深度、撮合频率与价格波动之间的定量关系。

Avellaneda--Stoikov 模型的核心思想在于:\emph{把做市问题看成在给定价格过程与订单流到达机制下,选择最优报价以最大化预期效用}。他们假设中间价遵循布朗运动,订单到达强度随报价偏离中间价的程度呈指数衰减,并采用指数效用(CARA)来度量风险厌恶。这样一来,做市商的最优报价可以通过求解相应的 HJB 方程得到,最终导出包含“风险补偿项”和“流动性补偿项”的显式最优价差公式\cite{AvellanedaStoikov2008}。在此基础上,Gueant 等人进一步引入库存上限与变换技巧,把原问题变为线性常微分方程组,并给出基于谱分析的近似闭式解\cite{GueantLehalleTapia2012},极大地方便了模型在真实交易系统中的落地与校准。

\section{加密货币 CEX 市场的特殊性}

尽管 Avellaneda--Stoikov 及其后续扩展模型为自动做市提供了坚实的理论基础,但这些模型最初主要针对股票或期货市场设计。加密货币 CEX(如 Binance, OKX, Bybit 等)在微观结构层面具有若干显著不同之处,这些差异要求在构建做市模型时进行有针对性的扩展与调整:

\begin{itemize}
  \item \textbf{7×24 小时连续交易与缺乏“收盘价”锚点}:传统市场的日内波动结构通常与开盘/收盘时段紧密相关,而加密市场无统一的日终清算时间,使得波动结构更为平滑但尾部风险更显著。
  \item \textbf{永续合约主导与 funding rate 反馈机制}:永续合约价格通过 funding 支付与现货价格锚定,funding rate 在一定程度上反映了多空力量的结构性失衡,并对价格 drift 产生反馈效应\cite{MakarovSchoar2021}。做市商在定价时不能简单忽略 drift。
  \item \textbf{多交易所碎片化与跨市场价格发现}:不同 CEX 之间价格高度相关但并不完全一致,跨交易所套利与做市天然交织,这要求在库存管理中同时考虑跨市场头寸与基差风险。
  \item \textbf{更强的高频波动与跳跃}:加密资产在宏观事件、链上清算、舆情信息等驱动下经常出现大幅跳跃(jumps),价格过程更适合用跳扩散模型刻画,这对做市模型提出更高要求。
  \item \textbf{费用结构与 rebate 机制}:Maker rebate 与 taker fee 的存在导致“名义价差”和“有效价差”并不一致,最优报价需要显式考虑手续费对收益与风险的影响。
  \item \textbf{技术基础设施的差异}:加密交易所 API 接口、撮合引擎与风控系统在稳定性和延迟方面差异显著,做市策略的实际表现很大程度上取决于系统工程与延迟管理。
\end{itemize}

这些特性意味着,简单地将传统做市模型照搬到加密货币 CEX 环境下往往无法取得满意效果。相反,需要在经典理论的基础上,结合订单流建模(如 Hawkes 过程\cite{BacryMuzyReview2015})、市场冲击模型\cite{Gatheral2010Impact,CarteaJaimungalPenalva2015Book}与最优执行理论\cite{AlmgrenChriss2001},构建适应加密微观结构的综合做市框架。

\section{本论文的研究问题}

在上述背景下,本论文围绕以下几个核心问题展开:

\begin{enumerate}[label=(Q\arabic*)]
  \item \label{Q:pricing} 在连续限价订单簿与随机订单流环境中,如何系统地推导最优 bid/ask 报价,使得做市商在风险厌恶偏好、库存约束与市场冲击约束下的效用最大化?
  \item \label{Q:inventory} 如何在数学上刻画库存风险,并将库存约束自然地嵌入最优做市模型中,从而得到既具有理论完备性又便于工程实现的库存管理策略?
  \item \label{Q:profitability} 在何种价格动态与订单流特性下,自动做市策略在长期具有正的期望收益?特别是在加密货币高波动环境中,怎样从理论上解释做市盈利的来源与边界?
  \item \label{Q:extension} 如何将经典做市模型扩展至更符合加密市场现实的设定,包括 drift、跳跃、Hawkes 订单流、非对称流动性、费用结构等?
  \item \label{Q:calibration} 在高频数据环境中,如何稳健地估计模型中的关键参数,如价格波动率、订单到达强度参数、冲击函数形状、Hawkes 核等,并检验模型与真实 LOB 的拟合程度?
  \item \label{Q:engineering} 如何基于上述理论构建一个在 BTC/ETH 现货与永续合约上可实盘运行的自动做市系统,包括定价引擎、订单管理系统、风险控制模块与仿真/回测框架?
\end{enumerate}

这些问题从理论、实证与工程三个维度共同构成本论文的研究主线。前几个问题偏重理论推导与模型构建,后几个问题偏重参数估计与系统实现。

\section{本论文的主要贡献}

围绕上述研究问题,本论文的主要贡献可以概括如下:

\begin{enumerate}[label=\textbf{C\arabic*.}]
  \item 在系统梳理市场微观结构与限价订单簿理论的基础上,给出一个统一的高频自动做市问题数学框架,将价格过程、订单流过程与做市商决策嵌入同一随机控制体系之中,明确区分并刻画了价差收益、库存风险与市场冲击三类核心要素的作用机制。
  \item 在 Avellaneda--Stoikov 模型\cite{AvellanedaStoikov2008}的基础上,详细推导了包含库存项的最优报价公式,并结合 Gueant--Lehalle--Tapia 的线性化方法\cite{GueantLehalleTapia2012},给出一套从 HJB 方程到线性 ODE、再到谱分解与闭式近似的完整推导路径,使该类模型在高维库存空间下仍然具备可计算性。
  \item 基于 Chakraborty--Kearns 对做市盈利性的分析\cite{ChakrabortyKearns2011},在更一般的价格模型(包括均值回复过程与含跳扩散过程)下,研究做市策略长期收益的条件,指出了在高波动但弱趋势的环境中做市更具优势,并将这一结论与加密货币市场的实证特征相对接。
  \item 将 Hawkes 过程、市场冲击模型与最优执行理论\cite{BacryMuzyReview2015,Gatheral2010Impact,CarteaJaimungalPenalva2015Book,AlmgrenChriss2001}引入做市框架,构建了适应加密 CEX 订单流特性的扩展模型,涵盖 drift、跳跃、非对称流动性与费用结构等现实因素。
  \item 提出一套面向高频数据的参数估计与模型校准方案,包括订单到达强度的极大似然估计、Hawkes 核的拟合、冲击函数的反演与波动率的高频估计,并系统讨论了加密货币数据中的噪音特征及其对估计的影响。
  \item 设计并实现了一个完整的 BTC/ETH 自动做市系统架构,包括定价引擎、订单管理系统、风险控制模块与 LOB 级别的仿真器,在真实市场数据与历史回测上验证理论模型的性能与局限,为未来在加密市场部署高频做市策略提供可复现的工程范式。
\end{enumerate}

\section{论文结构安排}

为了清晰呈现上述研究内容与贡献,本论文的结构安排如下:

\begin{itemize}
  \item 第 \ref{chap:intro} 章为引言,介绍研究背景、动机、核心问题与主要贡献,并对全文结构作出总体说明。
  \item 第 \ref{chap:microstructure} 章系统梳理限价订单簿与订单流的市场微观结构,包括 LOB 的形式化定义、订单类型、队列动态、价差与流动性度量,并结合加密货币 CEX 的具体制度与技术特征,为后续模型建立提供语义基础。
  \item 第 \ref{chap:math_foundations} 章介绍做市问题所需的数学基础,涵盖点过程、Hawkes 过程、SDE、随机控制与 HJB 方程,形成用于刻画价格过程与订单流过程的统一数学框架。
  \item 第 \ref{chap:as_model} 章详细推导 Avellaneda--Stoikov 做市模型,从基本假设出发建立 HJB 方程,给出最优报价的闭式解,并分析库存调整项与价差分解的经济含义。
  \item 第 \ref{chap:glt_model} 章重点研究 Gueant--Lehalle--Tapia 模型,通过变量变换将原问题线性化,并利用谱分解与渐近分析给出在有限库存约束下的最优报价近似,探讨其数值性质与工程可实现性。
  \item 第 \ref{chap:profit_ou} 章从长期盈利性角度分析做市策略在不同价格模型下的表现,特别关注均值回复过程与含跳过程,并将理论结果与加密货币市场的实证特征相联系。
  \item 第 \ref{chap:extensions} 章引入 drift、跳跃、Hawkes 订单流、市场冲击与费用结构等现代扩展,构建更贴近加密 CEX 实务环境的做市模型族。
  \item 第 \ref{chap:calibration} 章讨论参数估计与模型校准问题,包括订单到达强度、Hawkes 参数、冲击函数与波动率的估计方法,并在高频数据下评估模型拟合优劣。
  \item 第 \ref{chap:system_design} 章与第 \ref{chap:crypto_mm_framework} 章从工程系统角度出发,设计并实现一个适用于 BTC/ETH 现货与永续合约的自动做市系统,包括系统架构、实时定价、订单管理、风险控制与对冲策略等。
  \item 第 \ref{chap:experiments} 章利用仿真与历史回测,对所提出模型与系统进行数值实验分析,比较不同做市策略的表现,并探究参数敏感性与鲁棒性。
  \item 第 \ref{chap:conclusion} 章总结全文,梳理主要结论与贡献,并对未来研究方向进行展望。
\end{itemize}

通过上述结构安排,本论文力图在经典做市理论、加密货币微观结构与高频工程实践之间架起一座桥梁,为在加密货币限价订单簿市场中构建严谨而可落地的最优做市与库存管理体系提供系统化的理论与方法论基础。