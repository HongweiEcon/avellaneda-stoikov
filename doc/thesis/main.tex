% !TeX program = xelatex
\documentclass[12pt,a4paper,oneside]{book}

% =========================
% 编码与字体设置
% =========================
\usepackage{fontspec}
\usepackage{xeCJK}
\usepackage{amsmath,amssymb,amsthm,bm}
\usepackage{geometry}
\usepackage{graphicx}
\usepackage{booktabs}
\usepackage{enumitem}
\usepackage{url}
\usepackage{hyperref}
\usepackage{caption}
\usepackage{subcaption}
\usepackage{color,xcolor}
\usepackage{listings}
\usepackage{mathtools}
\usepackage{csquotes}
\usepackage{indentfirst}

% 字体设置
\setmainfont{Times New Roman}
\setsansfont{Helvetica Neue}
\setmonofont{Menlo}

\setCJKmainfont{PingFang SC}[AutoFakeBold=2.0,AutoFakeSlant=0.2]
\linespread{1.3}

% 页面设置
\geometry{
  a4paper,
  left=3cm,
  right=3cm,
  top=2.8cm,
  bottom=2.8cm,
  headsep=0.8cm,
  footskip=1.2cm
}

% 超链接
\hypersetup{
  colorlinks=true,
  linkcolor=blue,
  citecolor=teal,
  urlcolor=magenta
}

% 数学环境
\numberwithin{equation}{chapter}
\newtheorem{theorem}{定理}[chapter]
\newtheorem{proposition}[theorem]{命题}
\newtheorem{lemma}[theorem]{引理}
\newtheorem{corollary}[theorem]{推论}
\theoremstyle{definition}
\newtheorem{definition}[theorem]{定义}
\newtheorem{assumption}[theorem]{假设}
\newtheorem{remark}[theorem]{注}
\newtheorem{example}[theorem]{例}

%    常用数学符号
\newcommand{\E}{\mathbb{E}}
\newcommand{\Prob}{\mathbb{P}}
\newcommand{\Var}{\mathbb{V}\mathrm{ar}}
\newcommand{\R}{\mathbb{R}}
\newcommand{\diff}{\,\mathrm{d}}

% 代码环境
\lstset{
  basicstyle=\ttfamily\small,
  keywordstyle=\bfseries,
  commentstyle=\itshape,
  breaklines=true,
  frame=single,
  numbers=left,
  numberstyle=\tiny,
  tabsize=2,
  xleftmargin=2em,
  framexleftmargin=1.5em
}

% 参考文献
\usepackage[backend=biber,style=authoryear,maxbibnames=10,sorting=nyt]{biblatex}
\addbibresource{refs.bib}

% 封面信息
\newcommand{\ThesisTitle}{基于加密货币限价订单簿的最优做市理论、模型与系统实现}
\newcommand{\ThesisTitleEN}{Optimal Market Making Theory, Models and System Implementation in Cryptocurrency Limit Order Books}
\newcommand{\AuthorName}{私}
\newcommand{\AuthorNameEN}{Watashi}
\newcommand{\DegreeName}{深度研究报告}
\newcommand{\Supervisor}{(ChatGPT)}
\newcommand{\UniversityCN}{(纽约大学)}
\newcommand{\DepartmentCN}{经济学院}
\newcommand{\DateCN}{2025 年 11 月}

% =========================
% 文档开始
% =========================
\begin{document}

% 封面
\begin{titlepage}
  \centering
  \vspace*{2cm}
  {\Large \UniversityCN \par}
  \vspace{0.8cm}
  {\Large \DegreeName \par}
  \vspace{2.5cm}
  {\bfseries\LARGE \ThesisTitle \par}
  \vspace{0.8cm}
  {\large \ThesisTitleEN \par}
  \vspace{2.5cm}
  {\large 作者:\AuthorName\ (\AuthorNameEN)\par}
  \vspace{0.6cm}
  {\large 指导教师:\Supervisor\par}
  \vspace{2.5cm}
  {\large \DepartmentCN \par}
  \vspace{0.6cm}
  {\large \DateCN \par}
  \vfill
\end{titlepage}

% 前言部分
\frontmatter

% \chapter*{致谢}
% \addcontentsline{toc}{chapter}{致谢}

\chapter*{摘\ 要}
\addcontentsline{toc}{chapter}{摘\ 要}

\vspace{1em}
\noindent\textbf{关键词:} 自动做市;限价订单簿;加密货币;库存管理;高频交易

\chapter*{Abstract}
\addcontentsline{toc}{chapter}{Abstract}

\vspace{1em}
\noindent\textbf{Keywords:} market making; limit order books; cryptocurrency; inventory risk; high-frequency trading

% 目录
\tableofcontents
\clearpage
\listoffigures
\clearpage
\listoftables
\clearpage

% 正文开始
\mainmatter

% 第一层:市场微观结构基础
\chapter{引言}
\label{chap:intro}

\section{研究背景与动机}

过去二十年中,电子化与自动化交易极大地重塑了金融市场的微观结构。以限价订单簿(Limit Order Book, LOB)为核心的连续双边交易机制已经成为股票、期货、外汇等资产的标准交易模式\cite{BouchaudBook2010,ContStoikovTalreja2010}。在这一框架下,所有买卖意愿以限价单的形式在不同价格层级上排队等候成交,成交价格由主动提交的市价单(或价格跨越对手方报价的限价单)与订单簿上的挂单相互撮合而确定。市场参与者既可以作为“流动性提供者”(liquidity provider)通过挂单赚取价差,也可以作为“流动性需求者”(liquidity taker)通过吃单快速执行交易。

自动做市(automated market making)正是以限价订单簿为载体的一类关键策略:做市商通过持续在买一与卖一附近挂出双边报价,动态调节报价位置与挂单数量,为市场提供流动性,并从买卖价差中获取补偿。早在 Ho 和 Stoll 的经典研究\cite{HoStoll1981,HoStoll1983}以及 Glosten--Milgrom 模型\cite{GlostenMilgrom1985}中,做市商的核心问题就已经被形式化为:在面对信息不对称、库存压力与订单到达不确定性的前提下,如何设定买卖报价与库存调整策略,使得长期利润在风险约束下最大化。此后,Kyle 模型\cite{Kyle1985}进一步强调了信息驱动订单流对价格冲击和做市商盈亏的影响。

进入高频交易时代,做市问题的本质并未改变,但交易环境发生了质的飞跃:撮合速度以微秒计,订单流具有强烈的簇集性和自激性\cite{BacryMuzyReview2015},订单簿结构高度动态化,价格波动集中于极短时间尺度。Avellaneda 与 Stoikov 在 2008 年提出的做市模型\cite{AvellanedaStoikov2008}被广泛视为现代最优做市理论的起点:他们在连续时间、随机订单到达的框架下,将做市商的决策问题形式化为一类随机控制问题,并用 HJB(Hamilton--Jacobi--Bellman)方程刻画最优 bid/ask 报价与库存管理策略。随后,Gueant、Lehalle 与 Tapia 等人\cite{GueantLehalleTapia2012}对该模型进行重要扩展,通过引入适当的变量变换,将原本非线性的 HJB 方程转化为线性常微分方程组,使模型在工程上具备可实现性与可校准性。另一方面,Chakraborty 与 Kearns\cite{ChakrabortyKearns2011}从更加抽象的层面分析了做市策略长期盈利的来源,指出做市利润本质上取决于局部价格波动与长期价格趋势之间的权衡:只有当价格过程具有足够的短期波动而长期趋势不过于强烈时,做市才能在风险可控的前提下获得稳定的期望收益。

与此同时,以比特币(BTC)与以太坊(ETH)为代表的加密货币市场在过去十年间迅速扩张,衍生出以永续合约(perpetual swap)为主导的一整套交易生态系统。与传统股票或期货市场相比,加密货币市场具有若干鲜明特征:交易机制高度电子化、7×24 小时连续交易、现货与永续合约紧密联动、跨交易所价格发现过程复杂、清算机制与 funding rate 反馈影响价格路径等\cite{MakarovSchoar2021}。在这种环境中,自动做市不仅要处理传统意义上的库存风险与价格风险,还必须应对更强烈的波动性、更复杂的市场冲击与更显著的技术约束(如撮合延迟、带宽限制、风控指标实时监控等)。

因此,本论文的研究动机可以概括为:\emph{在加密货币限价订单簿这一高度动态且结构复杂的交易环境中,如何基于坚实的数学基础,构建一套理论上严谨、实践中可实现的最优做市与库存管理体系?}

\section{自动做市与限价订单簿的基本问题}

在限价订单簿框架下,做市商面临的基本决策包括:

\begin{itemize}
  \item \textbf{报价决策(quote placement)}:在当前中间价(mid-price)附近,选择多大的买卖价差(spread),以何种价格水平挂出 bid 与 ask?  
  \item \textbf{库存管理(inventory control)}:在订单随机成交的过程中,如何控制库存水平,使其在可接受的风险范围内波动?  
  \item \textbf{订单大小与挂单结构(order sizing \& layering)}:在各个价位挂出多少数量,是否采用多层挂单结构?  
  \item \textbf{订单生命周期管理(lifecycle management)}:多长时间刷新订单?在何种条件下撤单重挂?如何应对瞬时流动性枯竭或波动突增?  
  \item \textbf{风险与约束(risk constraints)}:如何在给定的风险限额、VaR/CVaR 约束与监管/风控指标下运行做市策略?  
\end{itemize}

传统微观结构理论将做市商面临的成本拆解为三部分:信息不对称导致的逆向选择成本(adverse selection cost)、库存成本(inventory cost)以及订单处理成本(order processing cost)\cite{GlostenMilgrom1985,Hasbrouck2007Book}。在高频环境中,这三类成本通过限价订单簿的形状、订单流的到达强度与撤单行为、以及价格冲击函数的形式共同体现出来。例如,Cont 等人\cite{ContStoikovTalreja2010}提出了一种随机 LOB 模型,将限价单、撤单与市价单到达均表示为点过程,从而可以在微观层面分析价差、深度、撮合频率与价格波动之间的定量关系。

Avellaneda--Stoikov 模型的核心思想在于:\emph{把做市问题看成在给定价格过程与订单流到达机制下,选择最优报价以最大化预期效用}。他们假设中间价遵循布朗运动,订单到达强度随报价偏离中间价的程度呈指数衰减,并采用指数效用(CARA)来度量风险厌恶。这样一来,做市商的最优报价可以通过求解相应的 HJB 方程得到,最终导出包含“风险补偿项”和“流动性补偿项”的显式最优价差公式\cite{AvellanedaStoikov2008}。在此基础上,Gueant 等人进一步引入库存上限与变换技巧,把原问题变为线性常微分方程组,并给出基于谱分析的近似闭式解\cite{GueantLehalleTapia2012},极大地方便了模型在真实交易系统中的落地与校准。

\section{加密货币 CEX 市场的特殊性}

尽管 Avellaneda--Stoikov 及其后续扩展模型为自动做市提供了坚实的理论基础,但这些模型最初主要针对股票或期货市场设计。加密货币 CEX(如 Binance, OKX, Bybit 等)在微观结构层面具有若干显著不同之处,这些差异要求在构建做市模型时进行有针对性的扩展与调整:

\begin{itemize}
  \item \textbf{7×24 小时连续交易与缺乏“收盘价”锚点}:传统市场的日内波动结构通常与开盘/收盘时段紧密相关,而加密市场无统一的日终清算时间,使得波动结构更为平滑但尾部风险更显著。
  \item \textbf{永续合约主导与 funding rate 反馈机制}:永续合约价格通过 funding 支付与现货价格锚定,funding rate 在一定程度上反映了多空力量的结构性失衡,并对价格 drift 产生反馈效应\cite{MakarovSchoar2021}。做市商在定价时不能简单忽略 drift。
  \item \textbf{多交易所碎片化与跨市场价格发现}:不同 CEX 之间价格高度相关但并不完全一致,跨交易所套利与做市天然交织,这要求在库存管理中同时考虑跨市场头寸与基差风险。
  \item \textbf{更强的高频波动与跳跃}:加密资产在宏观事件、链上清算、舆情信息等驱动下经常出现大幅跳跃(jumps),价格过程更适合用跳扩散模型刻画,这对做市模型提出更高要求。
  \item \textbf{费用结构与 rebate 机制}:Maker rebate 与 taker fee 的存在导致“名义价差”和“有效价差”并不一致,最优报价需要显式考虑手续费对收益与风险的影响。
  \item \textbf{技术基础设施的差异}:加密交易所 API 接口、撮合引擎与风控系统在稳定性和延迟方面差异显著,做市策略的实际表现很大程度上取决于系统工程与延迟管理。
\end{itemize}

这些特性意味着,简单地将传统做市模型照搬到加密货币 CEX 环境下往往无法取得满意效果。相反,需要在经典理论的基础上,结合订单流建模(如 Hawkes 过程\cite{BacryMuzyReview2015})、市场冲击模型\cite{Gatheral2010Impact,CarteaJaimungalPenalva2015Book}与最优执行理论\cite{AlmgrenChriss2001},构建适应加密微观结构的综合做市框架。

\section{本论文的研究问题}

在上述背景下,本论文围绕以下几个核心问题展开:

\begin{enumerate}[label=(Q\arabic*)]
  \item \label{Q:pricing} 在连续限价订单簿与随机订单流环境中,如何系统地推导最优 bid/ask 报价,使得做市商在风险厌恶偏好、库存约束与市场冲击约束下的效用最大化?
  \item \label{Q:inventory} 如何在数学上刻画库存风险,并将库存约束自然地嵌入最优做市模型中,从而得到既具有理论完备性又便于工程实现的库存管理策略?
  \item \label{Q:profitability} 在何种价格动态与订单流特性下,自动做市策略在长期具有正的期望收益?特别是在加密货币高波动环境中,怎样从理论上解释做市盈利的来源与边界?
  \item \label{Q:extension} 如何将经典做市模型扩展至更符合加密市场现实的设定,包括 drift、跳跃、Hawkes 订单流、非对称流动性、费用结构等?
  \item \label{Q:calibration} 在高频数据环境中,如何稳健地估计模型中的关键参数,如价格波动率、订单到达强度参数、冲击函数形状、Hawkes 核等,并检验模型与真实 LOB 的拟合程度?
  \item \label{Q:engineering} 如何基于上述理论构建一个在 BTC/ETH 现货与永续合约上可实盘运行的自动做市系统,包括定价引擎、订单管理系统、风险控制模块与仿真/回测框架?
\end{enumerate}

这些问题从理论、实证与工程三个维度共同构成本论文的研究主线。前几个问题偏重理论推导与模型构建,后几个问题偏重参数估计与系统实现。

\section{本论文的主要贡献}

围绕上述研究问题,本论文的主要贡献可以概括如下:

\begin{enumerate}[label=\textbf{C\arabic*.}]
  \item 在系统梳理市场微观结构与限价订单簿理论的基础上,给出一个统一的高频自动做市问题数学框架,将价格过程、订单流过程与做市商决策嵌入同一随机控制体系之中,明确区分并刻画了价差收益、库存风险与市场冲击三类核心要素的作用机制。
  \item 在 Avellaneda--Stoikov 模型\cite{AvellanedaStoikov2008}的基础上,详细推导了包含库存项的最优报价公式,并结合 Gueant--Lehalle--Tapia 的线性化方法\cite{GueantLehalleTapia2012},给出一套从 HJB 方程到线性 ODE、再到谱分解与闭式近似的完整推导路径,使该类模型在高维库存空间下仍然具备可计算性。
  \item 基于 Chakraborty--Kearns 对做市盈利性的分析\cite{ChakrabortyKearns2011},在更一般的价格模型(包括均值回复过程与含跳扩散过程)下,研究做市策略长期收益的条件,指出了在高波动但弱趋势的环境中做市更具优势,并将这一结论与加密货币市场的实证特征相对接。
  \item 将 Hawkes 过程、市场冲击模型与最优执行理论\cite{BacryMuzyReview2015,Gatheral2010Impact,CarteaJaimungalPenalva2015Book,AlmgrenChriss2001}引入做市框架,构建了适应加密 CEX 订单流特性的扩展模型,涵盖 drift、跳跃、非对称流动性与费用结构等现实因素。
  \item 提出一套面向高频数据的参数估计与模型校准方案,包括订单到达强度的极大似然估计、Hawkes 核的拟合、冲击函数的反演与波动率的高频估计,并系统讨论了加密货币数据中的噪音特征及其对估计的影响。
  \item 设计并实现了一个完整的 BTC/ETH 自动做市系统架构,包括定价引擎、订单管理系统、风险控制模块与 LOB 级别的仿真器,在真实市场数据与历史回测上验证理论模型的性能与局限,为未来在加密市场部署高频做市策略提供可复现的工程范式。
\end{enumerate}

\section{论文结构安排}

为了清晰呈现上述研究内容与贡献,本论文的结构安排如下:

\begin{itemize}
  \item 第 \ref{chap:intro} 章为引言,介绍研究背景、动机、核心问题与主要贡献,并对全文结构作出总体说明。
  \item 第 \ref{chap:microstructure} 章系统梳理限价订单簿与订单流的市场微观结构,包括 LOB 的形式化定义、订单类型、队列动态、价差与流动性度量,并结合加密货币 CEX 的具体制度与技术特征,为后续模型建立提供语义基础。
  \item 第 \ref{chap:math_foundations} 章介绍做市问题所需的数学基础,涵盖点过程、Hawkes 过程、SDE、随机控制与 HJB 方程,形成用于刻画价格过程与订单流过程的统一数学框架。
  \item 第 \ref{chap:as_model} 章详细推导 Avellaneda--Stoikov 做市模型,从基本假设出发建立 HJB 方程,给出最优报价的闭式解,并分析库存调整项与价差分解的经济含义。
  \item 第 \ref{chap:glt_model} 章重点研究 Gueant--Lehalle--Tapia 模型,通过变量变换将原问题线性化,并利用谱分解与渐近分析给出在有限库存约束下的最优报价近似,探讨其数值性质与工程可实现性。
  \item 第 \ref{chap:profit_ou} 章从长期盈利性角度分析做市策略在不同价格模型下的表现,特别关注均值回复过程与含跳过程,并将理论结果与加密货币市场的实证特征相联系。
  \item 第 \ref{chap:extensions} 章引入 drift、跳跃、Hawkes 订单流、市场冲击与费用结构等现代扩展,构建更贴近加密 CEX 实务环境的做市模型族。
  \item 第 \ref{chap:calibration} 章讨论参数估计与模型校准问题,包括订单到达强度、Hawkes 参数、冲击函数与波动率的估计方法,并在高频数据下评估模型拟合优劣。
  \item 第 \ref{chap:system_design} 章与第 \ref{chap:crypto_mm_framework} 章从工程系统角度出发,设计并实现一个适用于 BTC/ETH 现货与永续合约的自动做市系统,包括系统架构、实时定价、订单管理、风险控制与对冲策略等。
  \item 第 \ref{chap:experiments} 章利用仿真与历史回测,对所提出模型与系统进行数值实验分析,比较不同做市策略的表现,并探究参数敏感性与鲁棒性。
  \item 第 \ref{chap:conclusion} 章总结全文,梳理主要结论与贡献,并对未来研究方向进行展望。
\end{itemize}

通过上述结构安排,本论文力图在经典做市理论、加密货币微观结构与高频工程实践之间架起一座桥梁,为在加密货币限价订单簿市场中构建严谨而可落地的最优做市与库存管理体系提供系统化的理论与方法论基础。
\chapter{限价订单簿与加密货币 CEX 的市场微观结构}
\label{chap:microstructure}

\section{引言}

限价订单簿(Limit Order Book, LOB)是现代电子交易市场中价格形成的核心机制,负责记录市场参与者在不同时点对不同价格的买卖意愿。对于做市商而言,理解订单簿的动态结构、订单的到达与撤销规律、价差与深度的变化机制,是构建精确报价策略与库存管理方法的基础。

本章旨在从数学与经济学的角度,系统介绍 CEX(centralized exchanges)中 LOB 的定义、结构、动态,以及加密货币市场中特有的微观结构特征。内容将涵盖订单类型、队列优先级、订单流的随机结构、价格形成机制、微价格(microprice)、订单簿不平衡度(imbalance)、市场冲击与订单簿弹性等关键概念。我们将结合传统市场微观结构理论\cite{GlostenMilgrom1985,HoStoll1981,Kyle1985,Hasbrouck2007Book,BouchaudBook2010}以及针对加密货币市场的最新研究成果\cite{MakarovSchoar2021,DonierBouchaud2015Crypto},构建一个完整的 LOB 分析框架。

\section{限价订单簿的基本结构}

\subsection{订单簿的形式化定义}

一个标准的双边限价订单簿可以表示为一个时间依赖的价格–数量函数:
\[
\mathcal{L}_t = \{ (p, q(p,t)) : p \in \mathbb{R} \},
\]
其中 $q(p,t)$ 表示价格为 $p$ 时的挂单数量:

- $q(p,t) > 0$ 对应卖单量(ask side)
- $q(p,t) < 0$ 对应买单量(bid side)

在实际交易所中,价格是离散的(tick size $\Delta p$),因此订单簿由价格阶梯(price levels)构成:

\[
\mathcal{L}_t = \{(p_i, q_i(t)), i \in \mathbb{Z}\}.
\]

关键价格概念包括:

\[
\text{best bid: } b_t = \max\{p_i : q_i(t) < 0\},
\]
\[
\text{best ask: } a_t = \min\{p_i : q_i(t) > 0\},
\]
\[
\text{mid-price: } m_t = \frac{a_t + b_t}{2}.
\]

订单簿的即时卖买差(spread)为:
\[
s_t = a_t - b_t.
\]

CEX 市场的 tick size 通常固定,如 Binance 的 BTCUSDT tick size 为 0.1 USDT,而一些期货与 CEX perpetual 合约的 tick size 更小。

\subsection{订单类型与撮合机制}

交易所中存在三类基本订单:

\begin{itemize}
  \item \textbf{限价单(limit order)}:给出价格与数量,进入订单簿排队。
  \item \textbf{市价单(market order)}:立即与已有挂单撮合。
  \item \textbf{撤单(cancellation)}:从订单簿中移除。
\end{itemize}

订单簿的核心机制是 \textbf{价格优先、时间优先(price–time priority)}。对于做市商而言,这意味着:

- 提前挂单有更好的队列位置;
- 小幅提高或降低报价可显著提高成交概率;
- 每一次 cancel/replace 都会丢失队列位置,从而影响预期收益。

\subsection{撮合引擎与队列动态}

限价订单簿可以用三个主要的点过程描述:

- 限价单到达过程 $N^{LO}_t$,带有价格与数量属性;
- 市价单(或 crossing limit order)的到达过程 $N^{MO}_t$;
- 撤单到达过程 $N^{C}_t$。

Cont、Stoikov 与 Talreja\cite{ContStoikovTalreja2010} 提出了一个经典模型,将订单簿视为由一系列独立的 Poisson process 驱动的高维马尔可夫过程,能够解释:

- spread 的稳态分布;
- 深度分布的稳定性形态;
- 订单流不均匀导致的价格跳动。

在加密货币 CEX 中,上述过程更适合使用 Hawkes 过程\cite{BacryMuzyReview2015}建模,因为订单流具有显著的自激性,即大单消耗会引发更多同方向订单到达。

\section{价格形成机制}

\subsection{微价格(microprice)}

简单的 mid-price 不能完全反映短期价格预期,特别在订单簿不平衡时。Stoikov(2018)\cite{Stoikov2018Microprice} 提出“微价格”概念,用订单簿加权反映短期预期:

\[
\text{microprice}_t =
\frac{a_t \cdot V^{bid}_t + b_t \cdot V^{ask}_t}{V^{bid}_t + V^{ask}_t},
\]
其中 $V^{bid}_t$ 和 $V^{ask}_t$ 分别是买卖一档的挂单量。

若买方挂单更多,则 microprice 较接近 ask,表明短期上涨概率更高。

做市商可利用 microprice 调整 reservation price,从而提高 inventory 的动态控制效果。

\subsection{订单簿不平衡度(imbalance)}

经典定义为:

\[
I_t = \frac{V^{bid}_t - V^{ask}_t}{V^{bid}_t + V^{ask}_t}.
\]

实证研究(如 Huang–Stoll\cite{HuangStoll1994})显示,对高频价格变化具有显著预测力。

在加密 CEX 中,imbalance 的预测能力更强,因为:

- maker rebate 激励导致挂单量更可控;
- 大量高频参与者同步观察同一指标。

\section{订单流的统计特性}

\subsection{Poisson 与非齐次 Poisson 模型}

最基础的假设(AS 模型使用)是:

\[
\lambda(\delta) = A e^{-k\delta},
\]

反映报价远离 mid-price 时成交概率下降。

但是这一模型无法解释:

- 市价单爆发(bursts)
- order flow clustering
- volatility–volume coupling

\subsection{Hawkes 过程}

加密货币 CEX 中订单流呈现典型自激性。Bacry–Muzy\cite{BacryMuzyReview2015}模型定义如下:

\[
\lambda_t = \mu + \sum_{t_i < t} \phi(t-t_i),
\]

常用的指数核:

\[
\phi(t) = \alpha e^{-\beta t}.
\]

当 $\alpha/\beta \to 1$ 时,市场高度不稳定,出现 liquidity crisis 或流动性枯竭。

做市商的风险管理必须显式考虑这一点,因为:

- 强自激性意味着库存风险在高频尺度被放大;
- cancellation clustering 意味着队列位置变动频繁;
- burst 模式导致 spread 应在短期扩大。

\section{市场冲击与价格弹性}

市场冲击(market impact)描述市价单对未来价格的影响。Gatheral(2010)\cite{Gatheral2010Impact}指出,冲击应分为:

- permanent impact  
- transient impact  
- instantaneous impact  

冲击大小决定:

- maker 的 adverse selection 风险  
- 被动成交后价格向不利方向移动的概率  

做市商在报价时必须平衡:

- narrow spread → 高成交但高冲击风险  
- wide spread → 低成交但低风险  

这正是 AS-HJB 模型中最关键的 trade-off。

\section{加密货币 CEX 的微观结构特征}

\subsection{永续合约与 funding rate}

Makarov–Schoar(2021)\cite{MakarovSchoar2021}指出:

- perp price = spot price + basis  
- funding rate 驱动 perp price drift  

做市商在 perp 合约上必须将 drift 纳入 reservation price:

\[
r_t = m_t - q_t \gamma \sigma^2 + \text{funding\_adjustment}.
\]

\subsection{跨交易所价格发现与套利约束}

加密市场存在:

- 不同交易所的撮合延迟差异  
- 资金费率差异  
- 资金成本与风控规则差异  

做市商的库存管理必须考虑 cross-venue spread 和基差风险。

\subsection{延迟、架构与 queue-jump 风险}

CEX 市场没有统一撮合系统,不同交易所:

- 服务器架设位置不同  
- 撮合延迟不同  
- API 限速与播单(order stream)结构不同  

延迟越高,queue-jump 越严重,AS 模型中的到达强度需要进行 latency-aware 修正。

\section{本章小结}

本章从形式化定义、订单流、价格形成、市场冲击到加密货币 CEX 的特殊性,系统讨论了限价订单簿的微观结构。通过将传统微观结构理论与加密市场的独特机制相结合,我们为后续的做市模型、库存控制与工程系统搭建了语义与结构基础。

下一章将深入介绍做市问题的数学基础,包括点过程模型、Hawkes 过程、随机微分方程与 HJB 方程,为 Avellaneda–Stoikov 模型与后续扩展提供严谨的数学框架。

% 第二层:数学基础
\chapter{做市问题的数学基础}
\label{chap:math_foundations}

\section{引言}

自动做市的本质是一个典型的随机控制问题:在随机演化的价格过程与订单流过程中,做市商通过选择动态 bid/ask 报价与挂单数量,最大化某种风险调整后的目标函数。在这样的框架下,必须运用到随机微分方程(SDE)、点过程理论(Poisson 与 Hawkes 过程)、随机最优控制理论(HJB 方程),并构建市场微观结构层面的数学表述。

本章旨在为 Avellaneda--Stoikov 以及后续章节中的最优做市模型建立严格的数学基础。内容包括:价格过程的随机建模、订单流的点过程建模、效用函数与风险厌恶表示、库存动态的随机结构,以及 HJB 方程的推导与解释。

\section{价格过程的随机建模}

做市问题的第一层结构是价格动态模型。尽管真实市场中的价格表现包含跳跃、冲击与自激性,但在理论分析中,最基础的模型仍是布朗运动:

\[
\mathrm{d}S_t = \sigma \,\mathrm{d}W_t,
\]
其中 $W_t$ 为标准布朗运动。

这一模型具有以下优点:

\begin{itemize}
  \item 价格路径连续,避免了大量数学上的不连续项处理;
  \item $\sigma$ 代表瞬时波动率,是做市模型中最核心的参数;
  \item 与效用最大化框架结合可产生显式最优报价。
\end{itemize}

更一般的模型包括:

\subsection{含漂移项的 SDE}
加密永续合约中,由于 funding rate 引入长期漂移,合理的模型应包含:

\[
\mathrm{d}S_t = \mu_t \,\mathrm{d}t + \sigma_t \,\mathrm{d}W_t,
\]
其中 $\mu_t$ 可由 funding rate 估计(参见 Makarov--Schoar \cite{MakarovSchoar2021})。

\subsection{含跳跃的价格过程}

加密货币经常出现跳跃,可以用跳扩散模型:

\[
\mathrm{d}S_t = \sigma\, \mathrm{d}W_t + \kappa \,\mathrm{d}J_t,
\]

其中 $J_t$ 是泊松跳跃过程。跳跃会显著影响最优报价宽度,因为大跳风险增加做市的库存风险。

本论文后续将在第 \ref{chap:extensions} 章讨论跳扩散与做市的扩展模型。

\section{订单流的点过程模型}

做市商通过被动挂单赚取价差,因此成交的到达时间是核心变量。订单流在数学上通常用点过程(counting process)表示。

\subsection{Poisson 模型}

Avellaneda--Stoikov 模型假设成交是指数型随机过程,其到达强度与报价偏离中间价的距离相关:

\[
\lambda^{bid}(\delta) = A e^{-k \delta^b},\qquad
\lambda^{ask}(\delta) = A e^{-k \delta^a}.
\]

该假设具有以下意义:

\begin{itemize}
  \item 贴近 mid-price 的报价更容易成交;
  \item 参数 $k$ 表示订单流对价格的敏感度;
  \item 参数 $A$ 表示市场流动性规模;
  \item 强度的指数型结构使得最优报价可以显式求解。
\end{itemize}

这是做市模型的核心动力来源:spread 与成交概率之间的权衡。

\subsection{非齐次 Poisson 与扩展模型}

在更一般情形下,订单强度可以依赖价格状态、波动率或时间:

\[
\lambda_t = \lambda(S_t, \sigma_t, t).
\]

例如,在波动突增期,订单流会激增,做市商应扩大报价距离以降低风险。

\subsection{Hawkes 过程}

加密货币订单流具有强烈的自激性,Hawkes 过程是更适合的模型\cite{BacryMuzyReview2015}:

\[
\lambda_t = \mu + \sum_{t_i < t} \alpha e^{-\beta (t - t_i)}.
\]

其中:

- $\alpha/\beta$ 越大,自激性越强;
- 订单密集时成交风险更高;
- 做市商必须动态调整报价,避免过多 inventory 累积。

第 \ref{chap:extensions} 章将专门讨论 Hawkes 驱动的做市模型。

\section{现金与库存的动态}

做市商的状态变量包括现金账户 $X_t$ 与库存 $q_t$。  
当 bid 单成交时:

\[
q_t \to q_t + 1,\qquad
X_t \to X_t - (S_t - \delta_t^b).
\]

当 ask 单成交时:

\[
q_t \to q_t - 1,\qquad
X_t \to X_t + (S_t + \delta_t^a).
\]

因此总资产(wealth):

\[
W_t = X_t + q_t S_t.
\]

做市商的风险主要来自 $q_t S_t$ 的波动,即库存风险。  
库存越大,做市商越容易在价格跳动中亏损,这就是为什么做市模型必须包含库存处罚项(inventory penalty)。

\section{效用函数与风险厌恶}

Avellaneda--Stoikov 使用指数效用(CARA):

\[
U(w) = -e^{-\gamma w},
\]

其特点:

- 绝对风险厌恶度恒定;
- 加法结构使得 HJB 推导简单;
- 导致 reservation price 与 inventory 呈线性关系。

效用最大化目标:

\[
\max_{\delta^b,\delta^a} \mathbb{E}_t\left[
    -e^{-\gamma (X_T + q_T S_T)}
\right].
\]

这里 $\gamma$ 决定做市商的风险偏好:

- $\gamma$ 小 → 激进做市商 → spread 较窄;
- $\gamma$ 大 → 保守做市商 → spread 较宽。

\section{随机控制与 HJB 方程}

最优报价策略是一个连续时间随机控制问题。价值函数定义为:

\[
u(t, x, q, s) =
\max_{\delta^b, \delta^a}
\mathbb{E}\left[ -e^{-\gamma (X_T + q_T S_T)} \mid X_t = x, q_t = q, S_t = s \right].
\]

应用动态规划原理(Bellman optimality principle):

\[
0 = \partial_t u + \frac{1}{2}\sigma^2 \partial_{ss}u
+ \max_{\delta^b} \lambda^b(\delta^b) 
  \big[ u(t, x - (s-\delta^b), q+1, s) - u(t,x,q,s) \big]
\]
\[
+ \max_{\delta^a} \lambda^a(\delta^a) 
  \big[ u(t, x + (s+\delta^a), q-1, s) - u(t,x,q,s) \big].
\]

这是 Avellaneda--Stoikov 模型的核心 HJB。

简化后可导出:

- reservation price  
- optimal spread  

其推导将在下一章详细展开。

\section{库存风险与二次罚项}

另一类做市模型(如 Cartea--Jaimungal)采用对库存添加显式二次罚项:

\[
\text{Objective} = 
\mathbb{E}[W_T] - \frac{\gamma}{2}\mathbb{E}\left[ \int_0^T q_t^2 \sigma_t^2 \mathrm{d}t \right].
\]

意义:

- 直接惩罚大库存;
- 与布朗价格模型完美匹配;
- 适用于非指数效用的框架。

本论文仍采用 AS 模型的 CARA 效用,但库存罚项概念对工程系统非常重要(将在第 \ref{chap:system_design} 章讨论)。

\section{本章小结}

本章介绍了做市问题的数学基础,包括:

- 价格过程:布朗、含漂移、跳扩散;
- 点过程:Poisson 与 Hawkes;
- 现金与库存动态;
- CARA 效用与风险偏好;
- 随机控制与 HJB 方程框架;

这些工具构成后续 Avellaneda--Stoikov 模型(第 \ref{chap:as_model} 章)与 GLT 模型(第 \ref{chap:glt_model} 章)的基础。

下一章将从本章的基础数学出发,完整推导 Avellaneda--Stoikov 最优做市模型。

% 第三层:最优做市理论主干
\chapter{Avellaneda--Stoikov 最优做市模型}
\label{chap:as_model}

\section{引言}

Avellaneda--Stoikov(2008)模型是现代自动做市理论的基石,为做市商在随机价格与随机订单流环境下的最优 bid/ask 报价提供了一个可求解、结构清晰、经济含义强烈的动态模型。本章将从基本假设出发,结合上一章的数学基础,系统地推导最优报价、最优库存管理机制以及 reservation price(保留价格)与 optimal spread(最优价差)的闭式解,并解释其在加密货币 CEX 市场中的意义。

本章结构如下:

\begin{itemize}
  \item 第 \ref{sec:as_setup} 节介绍模型设定;
  \item 第 \ref{sec:as_intensity} 节介绍订单到达强度与流动性参数;
  \item 第 \ref{sec:as_hjb} 节推导核心 HJB 方程;
  \item 第 \ref{sec:as_solution} 节给出最优 bid/ask 的显式解;
  \item 第 \ref{sec:as_econ} 节解释解的经济含义;
  \item 第 \ref{sec:as_crypto} 节讨论在加密货币 CEX 环境中的适配;
\end{itemize}

本章提供了做市商与价格风险、库存风险之间权衡的数学基础,是后续带库存约束的 Gueant--Lehalle--Tapia 模型(第五章)的起点。

\section{模型设定}
\label{sec:as_setup}

做市商面对以下状态变量:

\begin{itemize}
  \item $S_t$: 中间价(mid-price)
  \item $q_t$: 库存(inventory)
  \item $X_t$: 现金账户(cash account)
\end{itemize}

总资产(wealth)为:
\[
W_t = X_t + q_t S_t.
\]

中间价遵循布朗运动:
\begin{equation}
\mathrm{d}S_t = \sigma\, \mathrm{d}W_t,
\end{equation}
其中 $\sigma$ 为瞬时波动率,$W_t$ 为标准布朗运动。

做市商在时间 $t$ 处提交两个限价单:

\[
S_t^b = S_t - \delta_t^b,\qquad 
S_t^a = S_t + \delta_t^a,
\]

其中:

- $\delta_t^b$:bid 偏离 mid-price 的距离;
- $\delta_t^a$:ask 偏离 mid-price 的距离;

做市商的决策变量是 $(\delta_t^b, \delta_t^a)$。

库存与现金的动态:

\[
q_t \to q_t + 1 \quad \text{(bid 成交)},
\quad 
X_t \to X_t - (S_t - \delta_t^b),
\]

\[
q_t \to q_t - 1 \quad \text{(ask 成交)},
\quad 
X_t \to X_t + (S_t + \delta_t^a).
\]

做市商希望最大化最终效用:

\[
\max_{\delta^b,\delta^a}
\mathbb{E}\left[ -e^{-\gamma (X_T + q_T S_T)} \right].
\]

其中 $\gamma > 0$ 是绝对风险厌恶度。

\section{订单到达强度模型}
\label{sec:as_intensity}

关键假设:成交强度随报价偏离 mid-price 的距离指数衰减(源自论文 \cite{AvellanedaStoikov2008}):

\begin{equation}
\lambda^b(\delta) = A e^{-k\delta},\qquad
\lambda^a(\delta) = A e^{-k\delta},
\label{eq:intensity}
\end{equation}

其中:
\begin{itemize}
  \item $A$:流动性规模;
  \item $k$:订单对价格敏感度;
\end{itemize}

经济直觉:

- $\delta$ 越小(接近 mid-price) → 成交概率高;
- $\delta$ 越大 → 成交概率低;
- $A,k$ 共同刻画 LOB 的深度与流动性条件。

后续 crypto 扩展模型会引入 bid/ask 非对称强度。

\section{HJB 方程推导}
\label{sec:as_hjb}

定义价值函数:

\[
u(t,x,q,s)=
\max_{\delta^b,\delta^a}
\mathbb{E}_t\left[ -e^{-\gamma(X_T + q_T S_T)} \right].
\]

使用动态规划原理与 Itô 引理,得 HJB 方程:

\[
0 = 
\partial_t u + \frac{1}{2}\sigma^2 \partial_{ss}u
+ 
\max_{\delta^b} 
\lambda^b(\delta^b)
\big[
u(t,x-(s-\delta^b), q+1,s) - u(t,x,q,s)
\big]
\]

\[
+
\max_{\delta^a} 
\lambda^a(\delta^a)
\big[
u(t,x+(s+\delta^a), q-1,s) - u(t,x,q,s)
\big].
\]

为求解此 PDE,我们利用 CARA 效用的指数结构:

\[
u(t,x,q,s) = -e^{-\gamma(x+qs)} \phi_q(t,s).
\]

代入,可消去 $x,s$ 项,并使问题只依赖 $q$。

进一步简化得到:

\[
\phi_q'(t) = -\frac{kA}{k+\gamma}
\left(
\phi_{q+1}(t) e^{-\gamma \delta_t^b}
+\phi_{q-1}(t) e^{-\gamma \delta_t^a}
\right).
\]

优化问题化为对 $\delta_t^b,\delta_t^a$ 的一阶条件求解。

\section{最优报价的解析解}
\label{sec:as_solution}

\subsection{reservation price}

令
\[
r_t = S_t - q_t \gamma \sigma^2 (T - t).
\]

这里 $r_t$ 是做市商在库存 $q_t$ 情况下的风险中性价格(indifference price)。

其含义:

- $q_t > 0$(多头) → reservation price 低于 mid-price → 倾向卖出;
- $q_t < 0$(空头) → 倾向买入;
- 库存越大,偏移越强。

\subsection{optimal spread}

AS 推导得最优价差:

\[
\delta_t^b + \delta_t^a
=
\gamma \sigma^2 (T - t)
+
\frac{2}{\gamma}
\ln\left(1 + \frac{\gamma}{k}\right).
\]

其中:

- 第一项是“风险价差”:库存风险越高,spread 越宽;
- 第二项是“流动性价差”:订单流越敏感($k$ 大),spread 越小。

\subsection{最优 bid/ask 价格}

最终得到最优 bid 与 ask:

\[
S_t^{b*}
=
r_t - \frac{1}{2}
\left(
\gamma\sigma^2 (T - t)
+
\frac{2}{\gamma}\ln\left(1+\frac{\gamma}{k}\right)
\right),
\]

\[
S_t^{a*}
=
r_t + \frac{1}{2}
\left(
\gamma\sigma^2 (T - t)
+
\frac{2}{\gamma}\ln\left(1+\frac{\gamma}{k}\right)
\right).
\]

这就是 Avellaneda--Stoikov 最经典的闭式解。

\section{经济含义与结构分析}
\label{sec:as_econ}

解的结构揭示了做市的三大经济力量:

\subsection{1. 库存风险机制}

reservation price:
\[
r_t = S_t - q_t \gamma \sigma^2 (T - t)
\]

是线性库存调整:

- 库存为正 → 降低 bid、抬高 ask;
- 库存为负 → 相反。

这是库存风险管理的数学主导机制。

\subsection{2. 价差分解}

最优价差可写为:

\[
\text{optimal spread}
=
\text{risk spread}
+
\text{liquidity spread}.
\]

解释:

- 风险越大 → 价差越宽;
- 流动性越高($k$越大)→ 最优价差越窄。

\subsection{3. 风险参数的意义}

- $\gamma$ 越大 → 做市商越保守;
- $\sigma$ 越大 → 做市越不安全,需扩大 spread;
- $k$ 越大 → 市场流动性更好,做市更容易。

\section{对加密货币 CEX 的适配}
\label{sec:as_crypto}

尽管 AS 模型起源于股票市场,但对 crypto CEX 的做市仍具有极强解释力。然而必须进行以下适配:
\begin{enumerate}
  \item 加入 drift(funding rate)\\
永续合约价格含有趋势项:  
funding rate $f_t$ 会使价格产生结构性 drift。

可扩展为:

\[
\mathrm{d}S_t = \mu_t \mathrm{d}t + \sigma \mathrm{d}W_t,
\quad
\mu_t = f_t \cdot S_t.
\]

\item 加入 bid/ask 非对称订单流\\
crypto CEX 常呈现非对称订单流:

\[
\lambda_t^b = A_b e^{-k_b \delta^b},\quad 
\lambda_t^a = A_a e^{-k_a \delta^a}.
\]

例如大户买盘推动、清算瀑布导致一边订单激增。

\item 加入费用结构(maker rebate)\\
maker rebate 会导致 “有效价差” 变化:

\[
\delta^{b,eff} = \delta^b - \text{rebate}.
\]

\item 加入跳跃与高波动区间报价\\
高波动期间应快速扩大 spread,可以把 $\sigma$ 换为实时估计的 $\hat{\sigma}_t$:

\[
\hat{\sigma}_t^2 = \text{realized volatility over past 1–5 seconds}.
\]
\end{enumerate}
\section{本章小结}

本章从最基本的市场设定出发,通过随机控制与指数效用框架,推导了 Avellaneda--Stoikov 最优做市模型的经典解析解,包括 reservation price、optimal spread 与动态报价机制。这一模型提供了做市商在库存风险、价格风险与订单到达概率之间权衡的数学基础。

下一章将基于此模型,引入库存上限与线性化技巧,推导现代工程上最可实现的 Gueant--Lehalle--Tapia 模型。
\chapter{Gueant--Lehalle--Tapia 模型:库存约束、线性化与谱分析}
\label{chap:glt_model}

\section{引言}

上一章给出了 Avellaneda--Stoikov(AS)最优做市模型的完整推导,并得到了 reservation price 与 optimal spread 的经典闭式解。然而,AS 模型存在以下两个局限:

\begin{enumerate}
  \item \textbf{库存无限($q$ 可任意大)}:实际交易中做市商受到严格的风险上限,例如 $|q| \le Q$;
  \item \textbf{HJB 非线性}:AS 的 HJB 方程在加入库存约束后变得更复杂,不易数值求解,更不易在高频系统中实时运行;
\end{enumerate}

为解决这些问题,Guéant、Lehalle 与 Tapia(GLT)在 2012 年的论文\cite{GueantLehalleTapia2012} 提出了一套极为重要的模型扩展,主要结果包括:

\begin{itemize}
  \item 引入库存上限(finite inventory buffer zone);
  \item 通过变量变换将非线性 HJB \emph{线性化};
  \item 建立关于库存 $q$ 的有限维线性 ODE 系统;
  \item 使用谱分解方法(spectral decomposition)求解;
  \item 基于极限情形给出 asymptotic closed-form solution;
\end{itemize}

GLT 模型是目前在实盘做市系统中最广泛使用的数学模型,具备可解释性、可实现性以及良好的数值稳定性。

本章将从 AS 模型出发,完整推导 GLT 模型的核心结果。

\section{有限库存约束与价值函数变换}

回顾 AS 模型的价值函数形式:

\[
u(t,x,q,s)= -e^{-\gamma(x+qs)} \phi_q(t).
\]

GLT 的关键 insight 是:当 $q$ 取有限值集合 $\{-Q,-Q+1,\dots,Q\}$ 时,可以将效用函数完全分离为:

\[
u(t, x, q, s)
=
-e^{-\gamma(x + qs)} 
v_q(t)^{-\gamma/k},
\quad q \in \{-Q,\dots, Q\}.
\]

其中:

- $v_q(t)$ 是新的未知函数;
- $k$ 是订单流敏感度(来自强度函数 $\lambda = A e^{-k\delta}$);
- $Q$ 是库存上限。

代入 HJB 后,可将非线性 PDE 消去,并得到关于 $v_q$ 的线性方程。

\section{线性 ODE 系统的推导}

代入 AS 强度:

\[
\lambda(\delta) = A e^{-k\delta},
\]

并对 $\delta_t^{b,a}$ 求最优解,可以得到最优价差只依赖于 $v_q$ 的比值:

\[
\delta^{b*}(q) 
= 
\frac{1}{k} 
\ln\left(\frac{v_q}{v_{q+1}}\right)
+
\frac{1}{\gamma}\ln\left(1+\frac{\gamma}{k}\right),
\]

\[
\delta^{a*}(q) 
= 
\frac{1}{k} 
\ln\left(\frac{v_q}{v_{q-1}}\right)
+
\frac{1}{\gamma}\ln\left(1+\frac{\gamma}{k}\right).
\]

将这些代回 HJB,即得:

\[
\dot{v}_q(t)
=
\alpha q^2 v_q(t)
-\eta 
\big(v_{q-1}(t) + v_{q+1}(t)\big),
\quad |q| < Q,
\]

边界条件:

\[
\dot{v}_Q = \alpha Q^2 v_Q - \eta v_{Q-1},
\quad
\dot{v}_{-Q} = \alpha Q^2 v_{-Q} - \eta v_{-Q+1},
\]

其中常数:

\[
\alpha = \frac{k}{2\gamma}\sigma^2,
\quad
\eta = A(1+\frac{\gamma}{k})^{-(1+k/\gamma)}.
\]

这样,原先高维、非线性、含位置、含现金的 HJB 问题,变成一个关于库存维度的 \textbf{$(2Q+1)$-维线性常微分方程组}:

\[
\dot{\mathbf{v}}(t) = M \mathbf{v}(t),
\quad
\mathbf{v}(T) = (1,1,\dots,1).
\]

其中矩阵 $M$ 为三对角矩阵:

\[
M_{q,q} = \alpha q^2,
\quad 
M_{q,q+1} = M_{q,q-1} = -\eta.
\]

这就是 GLT 模型的核心。

\section{谱分解求解}

矩阵 $M$ 是一个带状对称矩阵:

\[
M =
\begin{pmatrix}
\ddots & -\eta & 0 \\
-\eta & \alpha q^2 & -\eta \\
0 & -\eta & \ddots
\end{pmatrix}.
\]

由于 $M$ 是对称的,它拥有一组标准正交特征向量 $\{f^i\}$ 以及对应特征值 $\{\lambda_i\}$:

\[
M f^i = \lambda_i f^i.
\]

因此线性系统可解为:

\[
v_q(t)
=
\sum_{i=0}^{2Q}
c_i f^i_q e^{\lambda_i (t-T)},
\]

其中 $c_i$ 由终端条件 $v_q(T)=1$ 决定。

在实际实现中,$Q$ 通常不大(5–50),求解成本非常低。

\section{渐近闭式解(Asymptotic Closed Form)}

当 $Q$ 很大、$\eta/\alpha$ 处于适中规模时,可以证明最小特征值对应的特征向量渐近为:

\[
f^0_q \approx C e^{-\omega q^2},
\quad
\omega = \frac{1}{2}\sqrt{\frac{\alpha}{\eta}}.
\]

代回 bid/ask 价差公式,可得 asymptotic closed form:

\[
\delta^{b*}(q) 
\approx
\frac{1}{\gamma}
\ln\left(1+\frac{\gamma}{k}\right)
+
\frac{1}{2k}\sqrt{\frac{\alpha}{\eta}}(2q+1),
\]

\[
\delta^{a*}(q) 
\approx
\frac{1}{\gamma}
\ln\left(1+\frac{\gamma}{k}\right)
-
\frac{1}{2k}\sqrt{\frac{\alpha}{\eta}}(2q-1).
\]

与 AS 模型相比,这一版本:

- retain 完整库存非线性行为;
- 对大库存具有严格惩罚;
- 在高频系统中容易实时计算。

这是目前工程中最常用的做市公式之一。

\section{经济解释}

\subsection{库存越大,报价越偏离 mid-price}

bid/ask 的库存调整项:

\[
\pm \frac{1}{2k}\sqrt{\frac{\alpha}{\eta}}(2q \pm 1)
\]

说明:

- $q>0$(多头) → bid 往下调整、ask 往上调整;
- $q<0$(空头) → 反之。

库存越大,偏移越强。这与 AS 模型一致,但惩罚更加平滑。

\subsection{价差自动扩张}

最优 spread:

\[
\delta^{a*}(q)+\delta^{b*}(q)
=
\frac{2}{\gamma}
\ln\left(1+\frac{\gamma}{k}\right)
+
\sqrt{\frac{\alpha}{\eta}}.
\]

价差自动包含:

- 流动性惩罚 $\frac{2}{\gamma}\ln(1+\gamma/k)$;
- 库存惩罚 $\sqrt{\alpha/\eta}$。

\section{在加密货币 CEX 做市的适配}

GLT 模型在 crypto CEX 中比在传统股票市场更有价值:

\begin{itemize}
  \item 高波动性 $\sigma$ → 增强库存风险控制需求;
  \item 订单簿深度不稳定 → $A,k$ 经常要实时估计;
  \item 自激型订单流 → $\eta$ 可能动态变化;
  \item 多交易所报价 → $Q$ 需要动态调整;
\end{itemize}

在实际 BTC/ETH 做市系统中,通常使用:

- GLT asymptotic closed-form 用于高频更新;
- $A,k,\sigma$ 用 1–10 秒窗口实时估计;
- $Q$ 根据 vol、资金、风险限额动态调整。

\section{本章小结}

本章从 Avellaneda--Stoikov 模型出发,引入库存约束,并通过变量变换与线性化技巧,将做市问题转化为有限维线性 ODE 系统。通过谱分解可以求得解析形式的数值解,而渐近闭式解在工程系统中最为常用。

下一章将基于本章的库存结构,研究做市盈利的本质,并结合均值回复模型与加密货币价格行为分析长期做市收益。
\chapter{做市收益的来源与均值回复结构}
\label{chap:profit_ou}

\section{引言}

在前面几章中,我们基于 Avellaneda--Stoikov(AS)与 Guéant--Lehalle--Tapia(GLT)模型构建了最优做市的数学框架。然而一个更基础的问题是:

\begin{center}
\textit{做市为什么能赚钱?在什么条件下赚钱?在什么条件下会亏钱?}
\end{center}

这一问题不仅是理论上的,也是实际交易系统设计最核心的部分。在高频环境中,做市盈利主要来自三类因素:

\begin{enumerate}
  \item \textbf{spread capture(赚价差)}
  \item \textbf{inventory PnL(库存随价格变化带来的盈亏)}
  \item \textbf{order-flow prediction(短期价格预测)}
\end{enumerate}

其中第二项始终是负值(价格有噪声,库存引入风险),因此做市收益必须依赖于:

\[
\textbf{spread capture} > \textbf{inventory risk} + \textbf{adverse selection}.
\]

Chakraborty--Kearns(2011)\cite{ChakrabortyKearns2011} 提供了一个十分优雅的框架,把做市商的长期收益拆解为:

\begin{equation}
\mathbb{E}[\text{PnL}]
=
\underbrace{\mathbb{E}[\text{spread capture}]}_{\text{正收益}}
-
\underbrace{\mathbb{E}[q_t\, \mathrm{d}S_t]}_{\text{inventory risk}}
-
\underbrace{\text{adverse selection}}_{\text{信息不对称成本}},
\label{eq:pnl_decomp}
\end{equation}

并证明在一个均值回复价格过程下,做市策略可以长期正收益;而在强趋势市场中,做市必然亏损。

本章将从数学和经济学两方面深入分析做市盈利的来源。

\section{做市的 PnL 分解公式}

回顾总资产:

\[
W_t = X_t + q_t S_t.
\]

微分:

\[
\mathrm{d}W_t 
= \mathrm{d}X_t + q_t \mathrm{d}S_t.
\]

其中:

- $\mathrm{d}X_t$ 来自买卖价差(spread capture)
- $q_t \mathrm{d}S_t$ 来自库存风险(inventory PnL)

做市商的交易收益来自被动成交:

\[
\mathrm{d}X_t
=
(S_t + \delta_t^a)\,\mathrm{d}N_t^a
-
(S_t - \delta_t^b)\,\mathrm{d}N_t^b,
\]

其中 $\mathrm{d}N_t^{a,b}$ 是成交点过程。

取期望可得:

\[
\mathbb{E}[\mathrm{d}W_t]
=
\underbrace{
\delta_t^a \lambda_t^a + \delta_t^b \lambda_t^b
}_{\text{spread capture}}
+
\underbrace{
q_t\, \mu_t
}_{\text{inventory drift term}}.
\]

在均值为零的价格模型(如 $\mu_t=0$)下,第二项为零,但库存的方差导致风险成本:

\[
\mathrm{Var}[\mathrm{d}W_t] = q_t^2 \sigma^2.
\]

因此做市商必须控制库存波动,并以较大的价格偏移对冲风险,这正是 AS/GLT 模型中 reservation price 的来源。

\section{基于价格过程的长期盈利条件}

Chakraborty--Kearns(2011)做了一个重要发现:  
\textbf{做市商的长期收益完全取决于价格过程是否均值回复。}

他们考虑一个简单的做市策略:

- 每当价格上升到上轨,做市商卖出;
- 每当价格下降到下轨,做市商买入;
- 目标是赚取均值回复带来的回调。

假设价格满足 Ornstein--Uhlenbeck(OU)过程:

\[
\mathrm{d}S_t = \kappa(\theta - S_t)\,\mathrm{d}t + \sigma\,\mathrm{d}W_t.
\]

则做市商盈利的必要条件是:

\[
\kappa > 0.
\]

即价格对均值的回归速度为正。

进一步可得做市长期平均收益:

\[
\text{PnL} 
\approx 
2\Delta \cdot \kappa (S_t - \theta)
-
\frac{1}{2}\sigma^2 \Delta^2,
\]

其中 $\Delta$ 是做市商设置的对称挂单距离。

取期望,可得长期正收益条件:

\[
2\kappa \mathbb{E}[|S_t-\theta|] 
>
\frac{1}{2}\sigma^2 \Delta.
\]

该式意义明确:

- \textbf{均值回复越强} → 做市越赚钱;
- \textbf{波动越大} → 风险越大,spread 必须变宽;
- \textbf{挂单距离太宽} → 参与市场太少,spread capture 下降。

\section{为什么均值回复带来做市盈利?}

直觉如下:

- 做市商的库存倾向在价格上升时为负(卖出更多)
- 在价格下降时为正(买入更多)

因此库存与价格负相关:

\[
\text{Corr}(q_t, S_t) < 0.
\]

如果价格会回到均值,则:

- 高价卖出的库存会在低价回补 → 赚价差
- 低价买入的库存会在高价卖出 → 赚价差

这是 \textbf{inventory mean reversion}。

数学上:

\[
\mathbb{E}[q_t\, \mathrm{d}S_t] 
< 0
\quad\Longleftrightarrow\quad
\text{有盈利}.
\]

反之,如果价格有正趋势($\kappa <0$ 或 $\mu>0$):

\[
q_t \text{ 在趋势方向累积 } \Rightarrow \text{亏损}.
\]

这正解释了:

- crypto 上涨周期中做市商普遍亏损;
- 横盘震荡时期做市商利润极高。

\section{做市的收益结构:Adverse Selection 与 Price Impact}

完整 PnL 分解如下:

\begin{equation}
\mathrm{d}W_t
=
\underbrace{\text{spread capture}}_{(\delta^a\lambda^a + \delta^b\lambda^b)\,\mathrm{d}t}
\quad
-
\underbrace{q_t\,\mathrm{d}S_t}_{\text{inventory risk}}
\quad
-
\underbrace{\text{price impact}}_{\text{adverse selection}}.
\label{eq:pnl_full}
\end{equation}

其中 adverse selection 来自:

- 市价单方向 → 短期价格变化方向  
- 做市商被动吃单 → 往往在错误一侧成交  

若市价买单预示价格上行,则:

做市商的 ask 成交后  
→ 价格继续上涨  
→ 做市商亏钱。

做市要盈利必须要求:

\[
\text{spread capture}
>
\text{adverse selection}.
\]

而改进做法是提升预测水平:

- 若能预测订单流(例如 Hawkes 过程),可以调整挂单位置;
- 若能预测微价格(Stoikov 2018),可提前撤单或换档挂单。

\section{Crypto CEX 市场中的均值回复与做市盈利性}

加密货币市场展现了与传统股票不同的均值回复结构:

\subsection{短期均值回复强:}

来自:

- CEX order-book resiliency(Large 2007)
- 永续合约–现货基差收敛机制(funding)
- 内嵌的清算与杠杆机制
- 多交易所交叉影响(cross-exchange impact)

因此:

\[
\textbf{1-5 秒时间尺度上价格强烈均值回复}.
\]

这使得高频做市极其盈利。

\subsection{中长期趋势强:}

来自:

- 宏观资金面;
- BTC/ETH 作为风险资产的供需变化;
- 链上结构性因素(矿工、stakers)。

因此:

\[
\textbf{数小时–数天尺度上价格具有强趋势}.
\]

这解释了:

- 高频做市盈利稳定;
- 中频做市(如 5–15 分钟)常常因为趋势而亏损;
- 需要 GLT 模型强库存约束。

\section{做市盈利条件总结}

基于以上推导,做市盈利的必要条件可归纳为:

\begin{enumerate}
  \item \textbf{价格存在短期均值回复}  
      \[
      \kappa > 0 \quad\text{或}\quad
      \text{Corr}(q_t,\mathrm{d}S_t)<0.
      \]

  \item \textbf{订单流与价格没有强负向预测性}  
      (否则 adverse selection 过高)

  \item \textbf{库存风险可控}  
      (必须依赖 GLT 库存约束或动态对冲)

  \item \textbf{spread 足够覆盖价格噪声}  
      \[
      \delta > \sigma \sqrt{\mathrm{d}t}.
      \]

  \item \textbf{流动性足够高(成交概率高)}  
      ($A,k$ 估计稳定)
\end{enumerate}

在 crypto CEX 中,这五条都成立,因此做市长期盈利强。

\section{本章小结}

本章从 PnL 分解出发,揭示了做市盈利的本质来源:

\begin{itemize}
  \item spread capture(赚价差)是主要正收益;
  \item inventory risk 与 adverse selection 是主要负收益;
  \item OU 模型中的均值回复是盈利关键;
  \item Crypto CEX 在极短时间尺度具有非常强的均值回复,使得高频做市特别赚钱。
\end{itemize}

下一章将引入更多现实因素(drift、jump、Hawkes、手续费结构等),构建面向 Crypto 做市的扩展模型族。

% 第四层:现代做市扩展
\chapter{做市模型的扩展:订单流自激性、价格冲击、费用结构与跳跃}
\label{chap:extensions}

\section{引言}

在前几章中,我们重点讨论了 Avellaneda--Stoikov(AS)与 Guéant--Lehalle--Tapia(GLT)模型,它们在实际做市和学术界具有基础性地位。但这些模型的原始设定具有以下限制:

\begin{enumerate}
  \item 订单流为简单的 Poisson 模型,而真实订单流呈现自激性和簇集性;
  \item 价格模型为布朗运动,而 crypto CEX 中存在跳跃与冲击;
  \item 没有考虑手续费结构(maker rebate),实际收益偏差较大;
  \item 没有考虑永续合约特有的 funding rate 驱动的 drift;
  \item 库存管理没有考虑多交易所结构的基差风险;
  \item 没有加入订单簿状态的信息,例如 microprice 和 imbalance;
  \item 没有考虑批量成交、大单冲击与清算事件的影响;
\end{enumerate}

因此,本章将在 AS/GLT 框架基础上,引入更符合真实 crypto 市场的扩展模型族。

\section{带自激性订单流的 Hawkes 做市模型}

\subsection{Hawkes 过程回顾}

Bacry--Muzy(2015)\cite{BacryMuzyReview2015} 证明金融订单到达过程高度自激,可以表示为 Hawkes 过程:

\[
\lambda_t
=
\mu
+
\sum_{t_i<t}
\alpha e^{-\beta(t - t_i)}.
\]

其中:

- $\alpha$: 自激强度  
- $\beta$: 衰减速度  
- $\mu$: 基础流动性  

条件期望:

\[
\mathbb{E}[\lambda_t] = \frac{\mu}{1 - \alpha/\beta}.
\]

因此当 $\alpha/\beta$ 接近 1 时:

- 市场出现“流动性挤兑”(resiliency 下降)
- 做市风险急剧上升
- spread 应大幅扩张

\subsection{在做市 HJB 中加入 Hawkes 的方法}

对于 bid 与 ask 两侧:

\[
\lambda_t^b = A_b e^{-k_b \delta_t^b}
+
\sum_{t_i<t} \alpha_b e^{-\beta_b (t - t_i)},
\]

\[
\lambda_t^a = A_a e^{-k_a \delta_t^a}
+
\sum_{t_i<t} \alpha_a e^{-\beta_a (t - t_i)}.
\]

代入 AS 模型中的 HJB:

\[
\partial_t u + 
\frac{1}{2}\sigma^2 \partial_{ss} u
+
\max_{\delta^b} \lambda_t^b(\delta^b) \Delta u_{q+1}
+
\max_{\delta^a} \lambda_t^a(\delta^a) \Delta u_{q-1}
= 0.
\]

此处的主要修改是:

- 最优报价 $\delta^*$ 不再只依赖库存,还依赖 \(\lambda_t^b,\lambda_t^a\)
- 订单流自激性越强 → 越需要扩大报价距离以减少 adverse selection

这是 crypto CEX 做市中最重要的扩展之一。

\section{加入价格冲击(Market Impact)的做市模型}

\subsection{瞬时冲击与永久冲击}

Gatheral(2010)提出了 no-dynamic-arbitrage 的冲击结构:

\[
S_{t^+} = S_t + \underbrace{\eta v_t}_{\text{instantaneous}}
+ \underbrace{\int_0^\infty G(u)v_{t-u}\,\mathrm du}_{\text{transient}},
\]

其中:

- $v_t$ 为交易速度
- $\eta$ 为瞬时冲击系数
- $G(u)$ 为冲击核(impact kernel)

如果允许做市商主动调整库存(maker-taker hybrid),冲击不可忽略。

在做市框架中,只需修正:

- ask 成交通常意味着价格向上冲击  
- bid 成交通常意味着价格向下冲击  

因此 $\mathrm{d}S_t$ 应加入:

\[
\mathrm{d}S_t = \sigma \mathrm{d}W_t + \eta (\mathrm{d}N_t^a - \mathrm{d}N_t^b).
\]

带入 AS/GLT 模型,会发现:

- 最优 spread 必须随 $\eta$ 增大;
- 市价单方向越强 → 做市商越容易亏损 → 扩大价差;
- 使 adverse selection 提升为显性风险项。

\section{加入 maker rebate / taker fee 的有效价差模型}

Crypto CEX 通常有以下费用结构:

- maker:$r_m < 0$(负费 = rebate)
- taker:$f_t > 0$

做市商挂单成交后收益变为:

\[
\delta^{b,eff}
=
\delta^b + r_m,
\qquad
\delta^{a,eff}
=
\delta^a + r_m.
\]

因此 AS/GLT 中最优价差应替换为:

\[
\delta^{b,eff}(q), \quad \delta^{a,eff}(q).
\]

若 $|r_m|$ 较大(如 Binance 期货 VIP),可能出现:

- “zero-spread market making”
- 甚至 “negative spread arbitrage”(即补贴做市)

这是 crypto CEX 中做市利润较高的主要原因之一。

\section{加入跳跃过程(Jump Diffusion)}

Crypto 价格常出现跳跃:清算、监管、链上事件。

模型:

\[
\mathrm{d}S_t
=
\sigma \mathrm{d}W_t
+
\kappa \mathrm{d}J_t,
\quad
\mathbb{P}(\mathrm{d}J_t=1)=\lambda_J\,\mathrm{d}t.
\]

做市商在跳跃风险下应:

\begin{itemize}
  \item 扩大价差;
  \item 降低库存上限;
  \item 增大风险厌恶参数 $\gamma$;
\end{itemize}

跳跃会改变 AS 模型的 reservation price:

\[
r_t = S_t - q_t \gamma (\sigma^2 + \lambda_J \kappa^2)(T-t).
\]

库存风险变大。

\section{加入永续合约 funding rate 的漂移项}

永续合约价格近似满足:

\[
\mathrm{d}S_t = f_t S_t\,\mathrm{d}t + \sigma \mathrm{d}W_t.
\]

资金费率 $f_t$ 会带来方向性风险:

- $f_t > 0$ → 永续价格向上漂移 → 做市商容易积累空头亏损
- $f_t < 0$ → 永续价格向下漂移 → 做市商容易积累多头亏损

因此 reservation price 修正为:

\[
r_t
=
S_t 
- q_t \gamma \sigma^2 (T-t)
- q_t f_t S_t (T-t).
\]

如果 funding 高达年化 100–300\%,这项影响极大。

\section{跨交易所做市(Multi-Venue Market Making)}

做市商通常同时挂盘于:

- Binance Futures
- OKX Futures
- Bybit Futures
- Binance Spot
- Coinbase Spot

此时需要建模基差:

\[
B_t = S_t^{(1)} - S_t^{(2)}.
\]

基差具有极强均值回复(Makarov--Schoar 2021),因此做市商可利用:

- 多交易所 inventory balancing  
- 自然对冲(hedging across venues)  
- 更低的整体库存风险  

GLT 模型扩展为向量:

\[
\mathbf{q}_t = (q_t^{(1)}, q_t^{(2)},\dots)
\]

价值函数变为高维 ODE:

\[
\dot{v}_{\mathbf{q}}(t)
=
A(\mathbf{q}) v_{\mathbf{q}}
-
\sum_{i} \eta_i v_{\mathbf{q} + e_i}
-
\sum_{i} \eta_i v_{\mathbf{q} - e_i}.
\]

工程上通常采用低维投影(如 principal hedge factor)。

\section{加入 microprice / imbalance / queue 信息的信号驱动做市}

做市不应是完全被动的。

Stoikov(2018)提出 microprice:

\[
\text{microprice}_t =
\frac{a_t V_t^{bid} + b_t V_t^{ask}}{V_t^{bid} + V_t^{ask}}.
\]

若:

\[
\text{microprice}_t > m_t,
\]

价格有上涨预期,应适当:

- 撤销 ask 单;
- 缩小 bid–ask 方向性倾斜;
- 加速去库存;

可将其并入 AS 的 reservation price:

\[
r_t^{new}
=
r_t + \alpha (\text{microprice}_t - m_t).
\]

其中 $\alpha$ 是信号灵敏度。

此外,可以加入:

- queue size(排队深度)
- imbalance(订单不平衡度)
- Markov-switching volatility regime
- short-term alpha signals(如自回归、microstructure alpha)

这形成 “predictive market making”。

\section{本章小结}

本章构建了一个完整的扩展做市模型族,使得 AS/GLT 在真实 crypto CEX 微观结构中具有可行性。扩展包括:

\begin{itemize}
  \item Hawkes 自激订单流  
  \item 价格冲击与 adverse selection  
  \item maker rebate / taker fee  
  \item 跳跃风险  
  \item 永续合约 funding rate  
  \item 多交易所库存控制  
  \item 基于 microprice/imbalance 的 alpha 驱动做市  
\end{itemize}

这些扩展将在下一章中的校准与估计中发挥核心作用,为第 9–11 章的实盘系统实现提供数学基础。

% 第五层:参数估计与模型校准
\chapter{参数估计与做市模型的校准}
\label{chap:calibration}

\section{引言}

在前几章中,我们构建了做市商面临的数学框架,包括价格模型、订单流模型、库存动态以及最优控制方程。然而,要使做市策略在真实市场中运行,必须估计模型中的关键参数。

本章系统讨论以下参数的估计方法:

\begin{itemize}
  \item 价格波动率 $\sigma$
  \item 订单到达强度参数 $(A, k)$
  \item Hawkes 自激性订单流参数 $(\mu, \alpha, \beta)$
  \item 跳跃强度与冲击分布 $(\lambda_J, \kappa)$
  \item 永续合约的 funding drift $\mu_t$
  \item 市场冲击核 $G(t)$ 与瞬时冲击 $\eta$
\end{itemize}

同时,我们讨论如何将这些参数集成到做市系统,使其在高频数据流下实现实时更新。

\section{价格波动率 $\sigma$ 的高频估计}

\subsection{基于 mid-price 的 realized volatility}

Crypto CEX 的撮合频率极高,可以用成交价或 mid-price 构建:

\[
\hat{\sigma}_t^2
=
\sum_{i=t-\Delta t}^t (m_i - m_{i-1})^2.
\]

常用窗口:

- 高频做市:1–5 秒  
- 中频做市:30–300 秒  

为稳定估计,可使用 EWMA:

\[
\hat{\sigma}_t^2
=
\lambda \hat{\sigma}_{t-1}^2
+
(1-\lambda)(m_t - m_{t-1})^2.
\]

其中 $\lambda \approx 0.9$–$0.99$。

\subsection{带跳跃的波动率估计}

如果价格存在跳跃,应使用:

\[
\hat{\sigma}_t^2 = RV_t - \sum_{j \in \text{jumps}} (\Delta S_j)^2.
\]

跳跃可通过 bipower variation 判定:

\[
BV_t = \frac{\pi}{2}\sum |r_{i-1}||r_i|.
\]

当

\[
|r_i|^2 - BV_t > 3 \cdot \text{std}(BV)
\]

则视为跳跃。

\section{订单流强度参数 $(A, k)$ 的估计}

AS/GLT 强度模型:

\[
\lambda(\delta) = A e^{-k\delta}.
\]

对于成交事件 \((\delta_i,N_i)\),MLE 估计为:

\[
\ln L(A,k)
=
\sum_i \ln\left(Ae^{-k\delta_i}\right)
-
A\sum_i \int e^{-k\delta_i(t)} \mathrm{d}t.
\]

求导可得:

\[
\hat{k}
=
\operatorname{argmin}_k
\left(
\sum_i k\delta_i + A(k) T
\right).
\]

实际中更简单方法:

\[
\ln \lambda_i = \ln A - k\delta_i.
\]

拟合线性模型:

\[
y_i = \alpha + \beta x_i,
\quad
y_i=\ln \lambda_i,\, x_i=\delta_i.
\]

即可得:

\[
k = -\beta,\qquad A = e^\alpha.
\]

Crypto CEX 的经验:

- $k$ 可数秒内变化(市场紧张时变大)
- $A$ 在流动性事件中跳变(如清算)

因此应使用 rolling regression。

\section{Hawkes 自激性订单流参数 $(\mu,\alpha,\beta)$}

对买/卖方向分别估计:

\[
\lambda_t = \mu + \sum_{t_i<t} \alpha e^{-\beta(t-t_i)}.
\]

log-likelihood:

\[
\ln L
=
\sum_{i}
\ln \lambda_{t_i}
-
\int_0^T \lambda_t\, \mathrm{d}t.
\]

对应:

\[
\int_0^T \lambda_t\,\mathrm{d}t
=
\mu T
+
\sum_{t_i<t_j} \frac{\alpha}{\beta}
\left(
1-e^{-\beta(t_j-t_i)}
\right).
\]

可通过数值优化求解:

- BFGS
- L-BFGS-B
- Newton-Raphson

参数含义:

- $\alpha/\beta$ 越接近 1 → 订单流越有簇集 → 做市风险越大
- Crypto 中常见 $\alpha/\beta \in [0.6,0.9]$

可实时更新(每 1–5 秒)。

\section{市场冲击参数的估计}

采用 Gatheral (2010) 的冲击模型:

\[
\Delta S_t = \eta v_t + \sum G(u)v_{t-u}\mathrm du + \epsilon_t.
\]

估计方法:

- 对市场订单大小与即时价格变化回归
- 对 transient impact 进行核回归或 GMM

例如瞬时冲击:

\[
\Delta S_t = \eta q_t + \epsilon_t.
\]

用最小二乘估计:

\[
\hat{\eta}
=
\frac{\sum q_t \Delta S_t}{\sum q_t^2}.
\]

\section{跳跃参数 $(\lambda_J,\kappa)$ 的估计}

跳跃模型:

\[
\mathrm{d}S_t = \sigma \mathrm{d}W_t + \kappa \mathrm{d}J_t.
\]

跳跃检测可用:

- bipower variation
- threshold method

跳跃幅度 \(\kappa\) 用跳跃样本平均:

\[
\hat{\kappa} = \text{mean}(|\Delta S_{jump}|).
\]

跳跃强度:

\[
\hat{\lambda}_J = \frac{\text{jump count}}{T}.
\]

Crypto 市场跳跃频繁,因此模型应动态调整。

\section{永续 funding drift 的估计}

永续价格满足:

\[
\mathrm{d}S_t = f_t S_t \mathrm{d}t + \sigma\mathrm{d}W_t.
\]

实时 drift:

\[
\mu_t = f_t S_t.
\]

但 funding 通常 8 小时结算一次,因此短期影响可通过:

\[
\mu_t^{eff}
=
\mathbb{E}[f_{next}] \cdot S_t
\]

估计未来 funding 可用:

- Net basis
- Premium Index
- Orderbook imbalance
- Funding prediction ML model

\section{多交易所基差与相关性参数的估计}

若做市跨多个交易所:

\[
B_t = S_t^{(1)} - S_t^{(2)}.
\]

基差均值回复模型:

\[
\mathrm{d}B_t = -\kappa_B B_t \mathrm{d}t + \sigma_B \mathrm{d}W_t.
\]

估计方法:

- OLS 回归(equilibrium relationship)
- 2SLS 或 ECM(error correction model)

相关性矩阵:

\[
\Sigma_{ij} = \text{Corr}(\mathrm{d}S_t^{(i)}, \mathrm{d}S_t^{(j)}).
\]

用于库存聚合管理(GLT 扩展)。

\section{模型参数的实时更新机制}

Crypto 做市系统一般每 1 秒或每 5 秒刷新全部参数:

\begin{itemize}
  \item $\sigma$:1–5 秒 realized volatility
  \item $A,k$:rolling regression (过去 50–200 成交)
  \item Hawkes $(\mu,\alpha,\beta)$:moving MLE(30–60 秒窗口)
  \item funding drift:来自交易所 API,插值得到短期 drift
  \item impact $\eta$:以 1 分钟窗口回归更新
  \item jump intensity:rolling jump detection
\end{itemize}

参数刷新后:

- 更新 AS/GLT 最优报价;
- 更新库存 penalty;
- 更新风险限制;
- 重新生成 bid/ask 挂单。

这部分将在第 9–11 章的工程系统中完整实现。

\section{本章小结}

本章提供了做市模型校准的完整方法,包括:

\begin{itemize}
  \item 波动率估计(realized volatility / EWMA / jump-adjusted)
  \item 强度参数 $(A,k)$ 的 MLE 与回归估计
  \item Hawkes 参数的 log-likelihood 估计
  \item 冲击核、跳跃参数、funding drift 的估计方法
  \item 多交易所相关结构的估计
  \item 高频 trading 系统的实时参数更新方案
\end{itemize}

这些参数是将理论做市模型落地为可运行交易系统的关键。

下一章将进入实现层面:如何把这些模型嵌入真实高频做市系统。

% 第六层:工程落地体系
\chapter{自动做市系统架构设计}
\label{chap:system_design}

\section{引言}

在前八章中,我们从市场微观结构、随机控制理论、最优做市模型、扩展模型到参数估计,系统构建了完整的自动做市理论框架。然而,数学模型本身并不能直接在真实市场运行。要将其转化为稳定、高性能、低延迟、可风险控制的交易系统,需要严格的工程架构。

本章将设计一个可部署于加密货币 CEX(如 Binance、OKX、Bybit)的高频自动做市系统。系统由以下核心模块组成:

\begin{enumerate}
  \item 实时数据接入模块(Market Data Feed Handler)
  \item 定价引擎(Pricing Engine)
  \item 订单管理系统(Order Management System, OMS)
  \item 库存与风险控制系统(Risk Engine)
  \item 参数更新引擎(Parameter Refresh Engine)
  \item 报价与执行策略调度器(Strategy Scheduler)
  \item 仿真器与回测引擎(Simulator \& Backtester)
  \item 系统监控与日志框架(Monitoring \& Logging)
\end{enumerate}

这些模块必须协同运行,在毫秒级时间尺度下完成全链条计算与决策。

\section{系统总体架构}

图 \ref{fig:system_architecture}(请在最终论文中放图)展示了系统架构的总览,自上游的数据流到下游的订单执行路径清晰可见。

系统可抽象为三个主要层级:

\begin{itemize}
  \item \textbf{数据层(Data Layer)}:接收交易所原始数据(Orderbook、Trades、Funding、Index Price)
  \item \textbf{逻辑层(Logic Layer)}:定价、风险、策略、参数估计
  \item \textbf{执行层(Execution Layer)}:订单生成、撤单、重挂、成交处理
\end{itemize}

每一层都对系统稳定性、延迟和收益具有决定性影响。

\section{实时数据接入模块}

自动做市是一个强依赖市场微观结构的策略,因此“数据延迟”直接决定做市盈利能力。Crypto CEX 数据一般通过 WebSocket 接收:

\[
\text{Data} = \{ \text{Orderbook L2}, \text{Trade Prints}, \text{Ticker}, \text{Funding Rate}, \dots \}.
\]

要求:

\begin{itemize}
  \item L2 orderbook 更新粒度需达到毫秒级;
  \item Trade Prints 用于订单到达强度、Hawkes 参数与 volatility 估计;
  \item Funding/Index 用于永续 drift 修正;
  \item 所有数据需严格按时间戳排序;
\end{itemize}

数据模块输出:

\[
\mathcal{D}_t = \{m_t, b_t, a_t, \Delta m_t, \text{OB}_t, \text{trades}, \text{imbalance}, V_t^{bid},V_t^{ask}\}.
\]

这些输入会进入定价引擎与参数引擎。

\section{定价引擎(Pricing Engine)}

定价引擎是做市策略的数学核心,它使用:

\begin{itemize}
  \item AS/GLT 模型(或扩展模型)
  \item 实时波动率 $\hat{\sigma}_t$
  \item 强度参数 $(\hat{A}_t, \hat{k}_t)$
  \item 订单流 Hawkes 参数 $(\hat{\mu},\hat{\alpha},\hat{\beta})$
  \item 基差、funding drift
  \item microprice、imbalance、queue-length signals
\end{itemize}

\subsection{核心输出}

定价引擎每秒或每 200ms 输出:

\[
\delta_t^{b*}(q_t),\quad \delta_t^{a*}(q_t)
\]

\[
S_t^{b*} = S_t - \delta_t^{b*},\qquad
S_t^{a*} = S_t + \delta_t^{a*}.
\]

这些价格直接进入订单管理系统。

\subsection{GLT 模型在引擎中的实现}

因为 GLT 提供线性 ODE:

\[
\dot{v}(t) = M v(t),
\]

我们可以:

\begin{itemize}
  \item 离线求特征值/特征向量;
  \item 在线用向量外积直接计算 $v_q(t)$;
  \item 实时更新 $(A,k,\sigma)$;
  \item 对库存变化进行即时调整;
\end{itemize}

这使得 GLT 模型可以毫秒级计算。

\section{订单管理系统(OMS)}

OMS 是做市系统的执行核心,负责:

\begin{enumerate}
  \item 下单(Limit)
  \item 撮合响应(Fill / Partial Fill)
  \item 撤单(Cancel)
  \item 改单(Replace)
  \item 订单生命周期管理(Aging)
\end{enumerate}

\subsection{OMS 必须保证的性质}

\begin{itemize}
  \item 幂等性(Idempotency)
  \item 低延迟(sub-ms 内部处理)
  \item 可恢复性(断线自动恢复挂单)
  \item 与交易所 API 一致性(Orderbook Snapshot 同步)
\end{itemize}

\subsection{订单刷新策略}

典型的做市刷新周期:

- 100ms – 500ms 刷新全部报价  
- 20ms – 50ms 检查是否被市场跳过(queue position risk)  
- 若价格偏离 mid-price 超过阈值立即撤单  

高频做市中一个关键策略是:

\[
\textbf{Cancel aggressive, place conservative.}
\]

OMS 的性能直接影响策略的胜率。

\section{库存与风险控制系统(Risk Engine)}

风险引擎的主要目标是:

\[
|q_t| \le Q_t^{max}.
\]

库存上限 \(Q_t^{max}\) 可动态控制:

\[
Q_t^{max} = f(\sigma_t, \text{Hawkes intensity}, \text{liquidity}, \text{vol regime}).
\]

示例:

\[
Q_t^{max} = \frac{K}{\sigma_t},
\]

即波动越大,库存越小。

Risk Engine 的功能包括:

\begin{itemize}
  \item 实时计算库存风险 $q_t \sigma_t$
  \item 使用 GLT 的库存倾斜调整 ask/bid
  \item 监测 extreme fills、清算行情、跳跃风险
  \item 平仓机制(当库存过大时主动吃单)
\end{itemize}

必要时执行主动对冲:

\[
\text{Hedge} = -q_t S_t^{perp}.
\]

\section{参数更新引擎}

从第八章的校准结果可知,所有参数必须滚动估计:

\[
\Theta_t = \{\sigma_t, A_t, k_t, \mu_t,\alpha_t,\beta_t, \eta_t, f_t,\lambda_J\}.
\]

参数更新周期:

- 高频做市:200ms–1s  
- 中频做市:1–10s  

更新方式:

- EWMA(volatility)
- Rolling regression($k$)
- Hawkes MLE(每 1–5 秒)
- Funding drift 通过 exchange APIs

参数更新后,会立即刷新报价。

\section{策略调度器(Strategy Scheduler)}

Scheduler 是整个系统的 “大脑”,负责:

\begin{itemize}
  \item 控制策略执行顺序
  \item 限制过度撤单(anti-cancel-throttle)
  \item 控制挂单量(per-layer max)
  \item 监控 market state
\end{itemize}

典型 workflow:

\begin{enumerate}
  \item 读取最新市场数据  
  \item 更新参数  
  \item 调用 Pricing Engine 生成报价  
  \item 调用 Risk Engine 调整库存倾斜  
  \item 调用 OMS 执行 / 撤单 / 改单  
\end{enumerate}

所有过程必须在 50–200ms 内完成。

\section{高频仿真器(LOB Simulator)}

回测完全无法模拟真实 CEX 成交,因为:

- queue priority
- market order bursts
- orderbook resiliency
- Hawkes effect
- latency 影响成交事件

因此需要构建 LOB Simulator:

\[
\mathcal{S}: \text{Order stream} \to \text{Simulated fills}.
\]

使用以下数据:

- L2 snapshots
- incremental diff feed
- trade prints
- cancellation bursts

仿真器返回:

- 是否成交?
- 成交价格?
- queue position?
- partial fill 还是 full fill?

真实做市回测必须使用 LOB Simulator。

\section{系统监控}

核心监控指标:

\[
\text{PnL}_t,\quad q_t,\quad \text{spread},\quad \sigma_t,\quad \lambda_t^a,\lambda_t^b,
\]

以及:

- OMS 错误率
- API 延迟
- order sent / filled ratio
- cancel rate
- 市场跳跃检测

监控系统必须能在:

- 清算行情
- network outage
- market halt  

情况下安全关闭系统。

\section{系统延迟与硬件架构}

Crypto CEX API 延迟典型值:

- Binance Futures:5–20ms
- OKX Futures:10–25ms
- Coinbase:20–50ms

系统目标:

\[
\text{internal latency} < 5\text{ ms}.
\]

硬件:

- 单机 + 多线程
- Python + C++/Rust mixed
- affinity pinning(固定 CPU 绑定)
- 高频时需要 binlog-based deterministic replay

\section{本章小结}

本章从工程角度构建了一个完整的加密货币做市系统,包括:

\begin{itemize}
  \item 实时数据接入  
  \item GLT/AS 定价引擎  
  \item OMS 订单执行模块  
  \item 风险管理引擎  
  \item 滚动参数更新  
  \item 仿真器与回测系统  
  \item 系统监控与低延迟设计  
\end{itemize}

本章将理论构建与实际交易系统完美结合,为下一章的 整套 Crypto 做市框架(Spot + Perpetual + Multi-Venue) 提供基础。
\chapter{加密货币多市场(现货–永续–跨交易所)一体化做市框架}
\label{chap:crypto_mm_framework}

\section{引言}

加密货币市场具有高度复杂的微观结构特征:

\begin{itemize}
  \item CEX 现货(spot)与永续合约(perpetual)高度联动;
  \item 永续合约价格通过 funding rate 与现货锚定;
  \item 多交易所(Binance、OKX、Bybit)之间存在持续基差;
  \item 高频波动显著,订单簿深度不断变化;
  \item 市场存在跳跃与清算驱动的 “流动性瀑布”;
\end{itemize}

因此,一个成功的做市系统不应局限于单一市场,而必须构建:

\begin{center}
\textbf{Spot + Perpetual + Multi-Venue 的一体化做市框架}
\end{center}

本章从数学建模、策略逻辑与工程系统三个维度,构建可直接部署的综合做市结构。

\section{BTC/ETH 现货–永续价格结构}

定义:

\begin{itemize}
  \item $S_t^{spot}$:现货价格(Binance、Coinbase 等)
  \item $S_t^{perp}$:永续合约价格(Binance Futures、OKX Futures)
  \item $f_t$:永续 funding rate
  \item $I_t$:指数价格(index price)
\end{itemize}

永续合约理论上与现货锚定:

\[
S_t^{perp} = S_t^{spot} (1 + b_t),
\quad
b_t \approx \int f_t\,\mathrm{d}t.
\]

因此基差 $b_t$ 应满足均值回复:

\[
\mathrm{d}b_t = -\kappa_b b_t\,\mathrm{d}t + \sigma_b\,\mathrm{d}W_t.
\]

该模型由 Makarov–Schoar(2021)实证支持。

\subsection{对做市的启示}

基差均值回复意味着:

\begin{itemize}
  \item perp–spot 做市天然具有低库存风险;
  \item 可跨市场自对冲(spot/perp hedge);
  \item perp 上的 drift ≈ funding;
\end{itemize}

因此做市不应单独建模 spot/perp,而应构建 unified price:

\[
S_t = S_t^{spot},\qquad 
S_t^{perp} = S_t + \tilde{b}_t.
\]

\section{跨市场库存向量与风险结构}

定义库存向量:

\[
\mathbf{q}_t = 
\begin{pmatrix}
q_t^{spot} \\
q_t^{perp(1)} \\
q_t^{perp(2)} \\
\vdots \\
q_t^{perp(n)}
\end{pmatrix},
\quad
q_t^{spot}, q_t^{perp(i)} \in \mathbb{R}.
\]

定义风险资产价格向量:

\[
\mathbf{S}_t =
(S_t^{spot}, S_t^{perp(1)},\dots,S_t^{perp(n)}).
\]

组合价值:

\[
W_t = X_t + \mathbf{q}_t^\top \mathbf{S}_t.
\]

库存风险的方差:

\[
\mathrm{Var}(\mathrm{d}W_t)
=
\mathbf{q}_t^\top \Sigma_t\, \mathbf{q}_t,
\]

其中 $\Sigma_t$ 是 spot–perp–multi-venue 协方差矩阵。

基差均值回复的存在,使得:

\[
\mathrm{Corr}(S^{spot}, S^{perp}) \approx 0.99,
\]

因此风险可以通过 delta-neutral hedge 有效控制:

\[
q_t^{spot} + \sum_i q_t^{perp(i)} \approx 0.
\]

这构成了 multi-market 做市的基础。

\section{GLT 的高维扩展:多市场库存控制}

将 GLT ODE 从一维扩展到多维:

\[
\dot{v}_{\mathbf{q}}(t)
=
-\mathbf{q}^\top \Gamma\, \mathbf{q}\, v_{\mathbf{q}}
+
\sum_{i=1}^n \eta_i 
\big(
v_{\mathbf{q}+e_i}
+
v_{\mathbf{q}-e_i}
\big),
\]

其中:

- $\Gamma$ 是 inventory penalty matrix(由 $\Sigma_t$ 决定)
- $e_i$ 是第 $i$ 市场的单位库存增量
- $\eta_i$ 对应市场 $i$ 的流动性

高维 GLT 模型太大(状态空间指数级),实际中使用:

- **低秩近似(Low-rank)**
- **主对冲方向(Principal Hedge Direction)**
- **总库存近似(Aggregate Inventory Approximation)**

定义 aggregate inventory:

\[
q_t^{agg} = q_t^{spot} + \sum_{i=1}^n q_t^{perp(i)}.
\]

然后将库存风险简化为:

\[
\mathrm{Risk}(q_t^{agg}) \approx 
\gamma \sigma^2 (T-t) (q_t^{agg})^2.
\]

从而重新使用 GLT 的一维公式:

\[
\delta^{b,a*}(q_t^{agg})
=
\text{GLT}(q_t^{agg}, A, k, \sigma).
\]

这大幅简化计算,足以应对实际交易环境。

\section{多市场报价结构}

对 spot 与 perp 生成独立但相关的报价:

\[
S^{spot,b*} = S_t - \delta^{b*}(q_t^{agg}),
\quad
S^{spot,a*} = S_t + \delta^{a*}(q_t^{agg}),
\]

\[
S^{perp,b*} = S_t^{perp} - \delta^{b*}(q_t^{agg}) - \phi b_t,
\quad
S^{perp,a*} = S_t^{perp} + \delta^{a*}(q_t^{agg}) - \phi b_t,
\]

其中 $\phi$ 控制 perp–spot 基差预期修正(hedge ratio)。

此外永续合约还需加入 funding drift:

\[
r_t^{perp} = S_t^{perp} - q_t \gamma \sigma^2 (T-t) - f_t S_t (T-t).
\]

\section{跨交易所一体化做市}

对于多交易所现货与永续:

定义每个交易所的 mid-price:

\[
m_t^{(i)},\quad i=1,\dots,N.
\]

跨市场价差:

\[
\Delta_{ij} = m_t^{(i)} - m_t^{(j)}.
\]

若:

\[
|\Delta_{ij}| > \theta,
\]

则可以:

- 在 $i$ 交易所挂 ask
- 在 $j$ 交易所挂 bid

形成自然对冲。

库存向量变为:

\[
q_t = (q^{spot}_{Bin}, q^{spot}_{OKX}, q^{perp}_{Bin}, q^{perp}_{OKX},\dots).
\]

Risk Engine 控制:

\[
\sum_i q_t^{spot(i)} + \sum_j q_t^{perp(j)} \approx 0.
\]

\section{做市、对冲与套利的一体化调度器}

统一调度器执行:

\begin{enumerate}
  \item 更新市场数据(spot, perp, basis, funding)
  \item 更新参数($\sigma, A, k,$ Hawkes, funding drift)
  \item 计算 unified inventory $q_t^{agg}$
  \item 计算 GLT 基础报价
  \item 加入基差、funding、冲击等修正
  \item 多市场报价生成
  \item OMS 执行
  \item Risk Engine 检查是否需要对冲
\end{enumerate}

特别是跨市场对冲:

\[
\text{hedge size} = -q_t^{agg}.
\]

若 perp 流动性更好,则优先用 perp 对冲;若 spot 较稳定,可以 spot hedge。

\section{示例:BTC/ETH 做市全流程}

在实际交易中,该框架运行如下:

\subsection{步骤 1:数据更新}

- 获取 Binance/OKX/Bybit Perp L2
- 获取 Binance/Coinbase Spot L2
- 计算 mid-price、funding、basis

\subsection{步骤 2:参数更新}

- $\sigma_t$:1 秒 realized vol
- $A_t,k_t$:rolling regression
- Hawkes 参数:1 秒 MLE
- funding drift:exchange API

\subsection{步骤 3:库存与风险}

计算:

\[
q_t^{agg}
=
q_t^{spot}
+
q_t^{perp}.
\]

若:

\[
|q_t^{agg}| > Q_t^{max},
\]

则立即主动对冲。

\subsection{步骤 4:报价生成}

- 基于 GLT 输出基础 $\delta^*$  
- perp 报价加入 funding 修正  
- spot/perp 加入基差修正  

\subsection{步骤 5:订单执行}

通过 OMS:
\begin{itemize}
  \item cancel \& replace
  \item multi-layer quoting
  \item skip-switching(检测不利行情及时撤单)
\end{itemize} 

\section{本章小结}

本章构建了一个可在真实 crypto 市场部署的完整多市场做市框架。其特点包括:

\begin{itemize}
  \item Spot–Perp 联动报价模型  
  \item 基差均值回复与 unified inventory  
  \item 多交易所协同做市  
  \item 多维库存向量与风险控制  
  \item 资金费率驱动的 drift 修正  
  \item 可实时计算的 GLT-based 价格输出  
  \item 一体化调度器管理全链路  
\end{itemize}

下一章将使用仿真器与历史市场数据对该框架进行严格验证。
\input{chapters/ch11_experiments.tex}

\input{chapters/ch12_conclusion.tex}

% 参考文献
\clearpage
\printbibliography[heading=bibintoc,title={参考文献}]

% 附录
\appendix
\input{chapters/appendixA.tex}
\input{chapters/appendixB.tex}
\input{chapters/appendixC.tex}

\end{document}